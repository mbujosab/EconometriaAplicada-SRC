% Created 2023-11-12 dom 23:11
% Intended LaTeX compiler: pdflatex
\documentclass[11pt]{article}
\usepackage[utf8]{inputenc}
\usepackage[T1]{fontenc}
\usepackage{graphicx}
\usepackage{longtable}
\usepackage{wrapfig}
\usepackage{rotating}
\usepackage[normalem]{ulem}
\usepackage{amsmath}
\usepackage{amssymb}
\usepackage{capt-of}
\usepackage{hyperref}
\usepackage[spanish, ]{babel}
\usepackage[margin=0.5in]{geometry}
\usepackage[svgnames,x11names]{xcolor}
\hypersetup{linktoc = all, colorlinks = true, urlcolor = DodgerBlue4, citecolor = PaleGreen1, linkcolor = SpringGreen4}
\PassOptionsToPackage{hyphens}{url}
\usepackage{nacal}
\usepackage{framed}
\usepackage{listings}
\input{hansl.tex}
\lstnewenvironment{hansl-gretl}
{\lstset{language={hansl},basicstyle={\ttfamily\footnotesize},numbers,rame=single,breaklines=true}}
{}
\newcommand{\hansl}[1]{\lstset{language={hansl},basicstyle={\ttfamily\small}}\lstinline{#1}}
\lstset{backgroundcolor=\color{lightgray!20}, }
\author{Marcos Bujosa}
\date{\today}
\title{Lección 8}
\begin{document}

\maketitle
\tableofcontents

\clearpage

\section{Intervalos y regiones de confianza}
\label{sec:org79452d8}
\begin{center}
\begin{tabular}{ll}
Guión: & \href{https://github.com/mbujosab/Ectr/tree/master/Practicas/Gretl/scripts/EjPvivienda4.inp}{EjPvivienda4.inp}\\[0pt]
\end{tabular}
\end{center}

Cargue los datos de precios de casas \texttt{data3-1.gdt} del libro de
Ramanthan.
\begin{verbatim}
open data3-1
\end{verbatim}

\begin{itemize}
\item Estime el modelo de siempre y guárdelo como icono.
\begin{verbatim}
Modelo1 <- ols price const sqft --vcv
\end{verbatim}

\item Observe la matriz de covarianzas entre los parámetros estimados del
modelo de regresión. Hay covarianza entre los estimadores, ¿con qué
signo?

\item Calcule los intervalos de confianza de los parámetros beta
estimados: desde la ventana del modelo estimado siga los pasos
\textbf{\emph{\texttt{Análisis -> Intervalos de confianza para los coeficientes}}} o
bien directamente en un guión o la consola de Gretl aplique
directamente las expresiones vistas en clase:
\begin{verbatim}
scalar ns = 0.05  # nivel de signifacion

scalar i1 = $coeff(const) - critical(t, $df, ns/2) * $stderr(const)
scalar s1 = $coeff(const) + critical(t, $df, ns/2) * $stderr(const)

scalar i2 = $coeff(sqft)  - critical(t, $df, ns/2) * $stderr(sqft)
scalar s2 = $coeff(sqft)  + critical(t, $df, ns/2) * $stderr(sqft)

matrix IC = {i1, s1; i2, s2}
print IC
\end{verbatim}

\item Recuerde que los intervalos de confianza al 95\% nos sirven para
contrastar hipótesis al 5\% de significación. Piense qué valores
están en el umbral de ser rechazados según los intervalos obtenidos.

\item Dado el signo de la covarianza entre los estimadores, ¿qué relación
cabe esperar entre ellos?

\item Visualice la región de confianza de los parámetros: desde en la
ventana del modelo estimado siga los pasos \textbf{\emph{\texttt{Análisis -> Elipse de
  confianza}}} y seleccione ambos regresores para ver la elipse de
confianza. ¿Confirma sus expectativas?

\item Contrastemos algunas hipótesis compuestas. En particular, contraste
las distintas combinaciones de valores que están en el umbral de ser
hipótesis a rechazar individualmente (las correspondientes a las
esquinas del cuadrado que se ve en el gráfico del apartado
anterior). En la ventana del modelo \textbf{\emph{\texttt{Contrastes -> Restricciones
  lineales}}} y se introducen las restriciones; o bien ejecute el
código de más abajo.

¿Cuál es la conclusión respecto a la elipse de confianza en relación
a los contrastes de hipótesis de dos parámetros?
\begin{verbatim}
restrict Modelo1
 b[1] = i1
 b[2] = s2
end restrict

restrict Modelo1
 b[1] = s1
 b[2] = s2
end restrict

restrict Modelo1
 b[1] = i1
 b[2] = i2
end restrict

restrict Modelo1
 b[1] = s1
 b[2] = i2
end restrict

/* notese que el contraste individual H0: b[2]=i2 esta en el limite de ser rechazado al 5% 
   Pero si conjuntamente tambien se considera que b[1]=s1 entonces no se rechaza ni con un
   13% de sigificacion */
\end{verbatim}
\end{itemize}


\vspace{10pt}
\noindent
\subsection{Código completo de la práctica \texttt{EjPvivienda4.inp}}
\label{sec:org76d7c4c}
\vspace{10pt}
\lstinputlisting{scripts/EjPvivienda4.inp}
\clearpage


\section{Experimento de Montecarlo para los intervalos de confianza}
\label{sec:orgdc8d8e9}

\begin{center}
\begin{tabular}{ll}
Guión: & \href{https://github.com/mbujosab/Ectr/tree/master/Practicas/Gretl/scripts/samplinghouses5.inp}{samplinghouses5.inp}\\[0pt]
\end{tabular}
\end{center}

Cargue los datos de precios de casas \texttt{data3-1.gdt} del libro de
Ramanthan.  Este experimento de Montecarlo es una extensión a los ya
realizados con estos mismos datos.

\begin{itemize}
\item Defina la serie \texttt{x} con las superficies y la serie \texttt{y} con los
precios. Programe un bucle como las otras veces:
\begin{verbatim}
open data3-1
series x = sqft
series y = price
#set seed 3213798
loop 100 --progressive --quiet
    <<Estimación MCO>>
    <<Cálculo de los límites de los intervalos de confianza>>
    <<Comprobación de si los parámetros pertenecen al intervalo>>
    <<Estimación de la varianza de los errores>>
    <<Muestra de resultados y guardado en disco>>
endloop
\end{verbatim}
\end{itemize}

La serie de cálculos son los siguientes (todos dentro del bucle).

\begin{enumerate}
\item El primer bloque de cálculos simula el modelo con nuevas
perturbaciones, lo estima por MCO y guarda los betas estimados y
sus errores estándar:
\begin{verbatim}
series U  = randgen(n, 0, 39)
series ys = 52 + 0.14*x + U
ols ys const x
scalar b1 = $coeff(const)
scalar b2 = $coeff(x)
scalar s1 = $stderr(const)
scalar s2 = $stderr(x)
\end{verbatim}

\item Luego calculamos los intervalos de confianza al 95$\backslash$%
\begin{verbatim}
scalar c1L = b1 - critical(t,$df,.025)*s1
scalar c1R = b1 + critical(t,$df,.025)*s1
scalar c2L = b2 - critical(t,$df,.025)*s2
scalar c2R = b2 + critical(t,$df,.025)*s2
\end{verbatim}

\item Verificamos si los verdaderos valores pertenecen al intervalo estimado
\begin{verbatim}
scalar p1 = (52  >c1L && 52  <c1R)
scalar p2 = (0.14>c2L && 0.14<c2R)
\end{verbatim}

\item Al finalizar todas las cuentas, queremos que Gretl nos muestre los
estadísticos de los parámetros estimados, y el porcentaje de veces
que el intervalo contuvo a los parámetros, y que guarde todo lo
calculado en el fichero de datos \texttt{cicoeff.gdt}.
\begin{verbatim}
print b1 b2 p1 p2 
store cicoeff.gdt b1 b2 c1L c1R c2L c2R 
\end{verbatim}
\end{enumerate}

Por último abrimos el fichero \texttt{cicoeff.gdt}. Generamos una serie \texttt{B1}
constante e igual a valor del parámetro constante del modelo simulado (\(52\))
y otra \texttt{B2} igual a valor del parámetro de la pendiente del modelo
simulado (\(0.14\)); y finalmente generamos unos gráficos que muestren los
intervalos estimados en cada iteración y si incluyen o no al verdadero
valor del parámetro correspondiente.
\begin{verbatim}
open "cicoeff.gdt"
series B1 = 52
series B2 = 0.14
IntervaloConstante <- gnuplot c1L c1R B1 --time-series --with-lines --output="display"
IntervaloPendiente <- gnuplot c2L c2R B2 --time-series --with-lines --output="display"
\end{verbatim}

\vspace{10pt}
\noindent
\subsection{Código completo de la práctica \texttt{samplinghouses5.inp}}
\label{sec:org8b6cbcc}
\vspace{10pt}
\lstinputlisting{scripts/samplinghouses5.inp}
\clearpage


\section{Precio de casas unifamiliares (constante más tres regresores)}
\label{sec:orge6d7877}
\begin{center}
\begin{tabular}{ll}
Guión: & \href{https://github.com/mbujosab/Ectr/tree/master/Practicas/Gretl/scripts/Houses.inp}{Houses.inp}\\[0pt]
\end{tabular}
\end{center}


\begin{description}
\item[{Objetivos.}] Son tres:
\begin{itemize}
\item Asimilar la interpretación ``ceteris paribus'' de los coeficientes de un modelo de regresión.
\item Eliminar variables no significativas de un modelo.
\item Comparar el ajuste de dos modelos.
\end{itemize}

\item[{Los datos}] son los del ejemplo de clase junto con dos variables
adicionales: número de dormitorios (\texttt{bedrms}) y cuartos de baño
(\texttt{baths}).

\item[{Para empezar}] cargue los datos de la base de datos de Gretl desde
la pestaña del manual de Ramanathan
\begin{verbatim}
open data4-1
\end{verbatim}
\end{description}

\subsection{Tareas}
\label{sec:org47dda6c}

\begin{description}
\item[{Actividad 1}] Piense cuáles son los signos esperados de los
parámetros del siguiente modelo
\begin{displaymath}
   \VA{price}=
   \beta_1\VAindUno+
   \beta_2\VA{sqft}+
   \beta_3\VA{bedrms}+
   \beta_4\VA{baths}+
   \per.
 \end{displaymath}
  donde \VA{price} es el precio de una vivienda, \VA{sqft} su
superficie, \VA{bedrms} es su número de dormitorios y \VA{baths} su
número de cuartos de baño.

\item[{Actividad 2}] Con los datos de la muestra Ajuste dicho modelo de
regresión por MCO y guárdelo como un icono.
\begin{verbatim}
Modelo1 <- ols price 0 sqft bedrms baths 
\end{verbatim}

\item[{Actividad 3}] ¿Confirman los resultados su previsión respecto a los
signos de los parámetros?

\item[{Actividad 4}] El modelo estimado sugiere que una casa de \(1600\) pies al cuadrado,
con \(3\) habitaciones y \(2\) cuartos de baño tiene un precio esperado de
\begin{displaymath}
  129.062 + 0.154800\times(1600) -21.5875\times(3) -12.1928\times(2) =  287.59 \; \text{miles de dólares.}
\end{displaymath}

Según este modelo (y sabiendo que un pie cuadrado son \(0.092m^2\))
\emph{ampliar} esta casa con una habitación adicional de unos \(20\) metros
cuadrados (unos \(220\) pies cuadrados más de casa) arrojaría un
precio esperado de ¿cuanto?
\begin{displaymath}
   129.062 + 0.154800\times(1820) -21.5875\times(4) -12.1928\times(2) =  300.06;
\end{displaymath}
es decir, \emph{ampliar} esta casa con una habitación adicional de unos
\(20\) metros cuadrados \emph{eleva} el precio en unos \(13\) mil dolares.
¿Contradice esto su idea inicial?
\begin{verbatim}
series yhat1 = $coeff(const)+$coeff(sqft)*1600+$coeff(bedrms)*3+$coeff(baths)*2
series yhat2 = $coeff(const)+$coeff(sqft)*(1600+220)+$coeff(bedrms)*(3+1)+$coeff(baths)*2
\end{verbatim}

\item[{Actividad 5}] Aunque, con la correcta interpretación de los
coeficientes, el resultado parece sensato, sólo una de las variables
es estadísticamente significativa.

\begin{itemize}
\item Esto quiere decir, que la estimación de los parámetros es muy
imprecisa. Sin embargo, el estadístico \(\Festadistico\) indica que
el modelo es conjuntamente significativo (así que la previsión de
precios de la actividad anterior es fiable); pero no el efecto
individual de cada regresor.

\item Observe además que hay un elevado grado de correlación entre los
regresores (en la ventana de iconos, ``pinche'' en
\textbf{\emph{\texttt{Correlaciones}}}) o ejecute los siguiente
\begin{verbatim}
corr sqft bedrms baths
\end{verbatim}
Esto sugiere que pudiera surgir un problema de multicolinealidad.
Además, la interpretación caeteris paribus cuestiona la relevancia
de algunos los regresores. Todo ello apunta a la posible
conveniencia de excluir uno o más regresores del modelo inicial.
\end{itemize}

\item[{Actividad 6}] Puesto que la variable menos significativa es
\texttt{baths}, ésta será la primera variable a omitir del modelo inicial:
\begin{itemize}
\item En la ventana del Modelo 1, ``pinche'' en \textbf{\emph{\texttt{Contrastes ->
    omitir variables}}} y seleccione la variable \texttt{baths} (y pulse
\textbf{\emph{\texttt{Aceptar}}}). O bien ejecute el código
\begin{verbatim}
omit baths
\end{verbatim}
Note como el coeficiente de determinación ha disminuido, pero el
corregido ha aumentado.
\end{itemize}

Aunque ha aumentado la significatividad de los coeficientes,
\texttt{bedrms} continua siendo no significativa. Así que la vamos a
omitir:
\begin{itemize}
\item En la ventana del Modelo 1, ``pinche'' en \textbf{\emph{\texttt{Contrastes ->
    omitir variables}}} y seleccione la variable \texttt{bedrms} (y pulse
\textbf{\emph{\texttt{Aceptar}}}). O bien ejecute el código
\begin{verbatim}
omit bedrms
\end{verbatim}
Note como de nuevo el coeficiente de determinación ha
disminuido, pero el corregido ha aumentado.
\end{itemize}

\emph{La constante sigue sin ser significativa ¿debemos hacer como con
\texttt{baths} y \texttt{bedrms} y tratar de omitirla también?}

\item[{Actividad 7}] Hemos excluido las variables \texttt{baths} y \texttt{bedrms} del
modelo original, debido a que individualmente son no significativas.

Pero pudiera ocurrir que conjuntamente si fueran
significativas. Vamos a verificar que conjuntamente tampoco son
significativas:

\begin{itemize}
\item En la ventana del Modelo 1, ``pinche'' en \textbf{\emph{\texttt{Análisis -> Elipse de
    confianza}}} y seleccione \texttt{baths} y \texttt{bedrms}.

\item Compruebe que \((0,0)\) es un punto que pertenece a la elipse de
confianza, y que por tanto, la hipótesis nula \(\beta_3=0,\;
    \beta_4=0\) no se puede rechazar al nivel de confianza de la
elipse.

\item Omita de golpe las dos variables: En la ventana del Modelo 1,
``pinche'' en \textbf{\emph{\texttt{Contrastes -> omitir variables}}} y seleccione
las variables \texttt{baths} y \texttt{bedrms} (y pulse \textbf{\emph{\texttt{Aceptar}}})
\begin{verbatim}
ols price 0 sqft bedrms baths
Modelo2 <- omit baths bedrms
\end{verbatim}

\item Observe los resultados.
\end{itemize}
\end{description}


\begin{description}
\item[{Actividad 8}] Comparar las previsiones del modelo ampliado y el
reducido.

Vamos a comprobar que las predicciones no difieren demasiado entre
uno y otro modelo. Pero hay mayor precisión con el modelo reducido
(no multicolinealidad).

\emph{Primero añadimos una observación adicional} con los datos de
superficie, número de dormitorios y cuartos de baño del ejemplo
anterior.
\begin{itemize}
\item ``Pinche'' en \textbf{\emph{\texttt{Datos -> Añadir observaciones}}} y añada una observación

\item Marque con el ratón las variables \texttt{sqft},
\texttt{bedrms} y \texttt{baths}\}, y con las tres variables
marcadas, pinche en \textbf{\emph{\texttt{Datos -> Editar valores}}}. Teclee \(1820\)
en la fila \(15\) de la columna \texttt{sqft}, teclee \(4\) en la fila
\(15\) de la columna \texttt{bedrms} y \(2\) en la fila \(15\) de la columna
\texttt{baths}.

\item Fije la muestra con las nuevas observaciones: ``pinche'' en
\textbf{\emph{\texttt{Muestra -> Establecer rango}}} y fíjelo para las observaciones
de 1 a 15.  O bien teclee en la  consola de comandos
\begin{verbatim}
dataset addobs 1
genr sqft[15] =1820
genr bedrms[15]=4
genr baths[15] =2
\end{verbatim}

\item Re-estime el primer modelo pero usando la muestra ampliada.
\begin{verbatim}
smpl 1 15
ols price const sqft bedrms baths
\end{verbatim}

\begin{itemize}
\item En la ventana del modelo re-estimado ``pinche'' en \textbf{\emph{\texttt{Análisis
      -> Predicciones}}}. Nos aseguramos de que el dominio de
predicción contiene la observación 15 y pulsamos \textbf{\emph{\texttt{Aceptar}}}.
\begin{verbatim}
fcast 15 15 --static
fcast --plot="display"
\end{verbatim}

\item Observe la previsión puntual y el intervalo de confianza
\end{itemize}

\item Re-estime también el segundo modelo con la muestra ampliada
\begin{verbatim}
ols price const sqft
\end{verbatim}

\begin{itemize}
\item En la ventana del modelo re-estimado ``pinche'' en \textbf{\emph{\texttt{Análisis
      -> Predicciones}}}. Asegúrese de que el dominio de predicción
contiene la observación 15 y pulse \textbf{\emph{\texttt{Aceptar}}}.
\begin{verbatim}
fcast 15 15 --static
fcast --plot="display"
\end{verbatim}
\end{itemize}

\item Note cómo ambas previsiones puntuales están contenidas en
ambos intervalos, por lo que no son estadísticamente distintas.

\item Pero fíjese en cómo la falta de precisión del primer modelo se
plasma en una mayor desviación típica, y en un intervalo de
confianza más amplio.
\end{itemize}
\end{description}
\subsection{Código completo de la práctica \texttt{Houses.inp}}
\label{sec:orga940e3e}
\lstinputlisting{scripts/Houses.inp}
\clearpage


\section{Los determinantes del número de viajeros de autobús}
\label{sec:orgbc38b3f}
\begin{center}
\begin{tabular}{ll}
Guión: & \href{https://github.com/mbujosab/Ectr/tree/master/Practicas/Gretl/scripts/BusTravelers.inp}{BusTravelers.inp}\\[0pt]
\end{tabular}
\end{center}

Aplicación 4.6 del libro de texto (Ramanathan, R., 1998). Para más
detalles consulte el citado manual.

\begin{description}
\item[{Objetivo}] Especificación de un modelo de regresión.

En general, eliminar variables no significativas incrementa la
precisión de la estimación del resto de parámetros. Pero eliminar de
golpe todas las variables no significativas no es recomendable. Al
quitar algunas variables puede que otras se vuelvan
significativas. Un procedimiento más cauteloso es ir quitando
variables de una en una.

La significación estadística no es el único criterio, ni el más
conveniente, para tomar una decisión. Si hay razones teóricas para
mantener una variable debemos cuestionarnos seriamente el quitar
dicha variable por el mero hecho de que ésta resulte no
significativa estadísticamente.

Por otra parte, siempre debemos mantener el término constante.

\item[{Los datos y el modelo}] Ésta es una aplicación con datos reales que
relaciona el número de horas viajadas en autobús (en miles)
\texttt{Bustravl} (\VA{B}) en 40 ciudades de los Estados Unidos con las
siguientes variables explicativas:
\begin{itemize}
\item \texttt{Fare} = la tarifa del billete en dólares (\VA{F}).
\item \texttt{Gasprice} =Precio del galón de gasolina en dólares (\VA{G}).
\item \texttt{Income} = Renta percápita media de la ciudad en dólares (\VA{I}).
\item \texttt{Pop} = Población de la ciudad en miles (\VA{P}).
\item \texttt{Density} = Densidad de población de la ciudad (personas por milla
cuadrada) (\VA{D}).
\item \texttt{Landarea} = Extensión del área urbana en millas cuadradas (\VA{L}).
\end{itemize}

La especificación inicial del modelo es:
\begin{displaymath}
  \VA{B}=
  \beta_1\VAindUno+
  \beta_2\VA{F}+
  \beta_3\VA{G}+
  \beta_4\VA{I}+
  \beta_5\VA{P}+
  \beta_6\VA{D}+
  \beta_7\VA{L}+
  \per.
\end{displaymath}
Se trata de decidir cuales de todas las variables disponibles es
razonable mantener en un modelo final.

\item[{Para empezar}] cargue los datos de la base de datos de Gretl desde
la pestaña del manual de Ramanathan.
\begin{verbatim}
open data4-4
\end{verbatim}
\end{description}

\subsection{Tareas}
\label{sec:org0b524f9}

\begin{description}
\item[{Actividad 1}] Piense cuáles son los signos esperados de los parámetros del modelo

\item[{Actividad 2}] Observe los diagramas de dispersión entre la variable
dependiente y los regresores. Observe así mismo los estadísticos
descriptivos de las variables y la matriz de correlaciones. Hágalo
con los menus desplegables o ejecutando el siguiente código:
\begin{verbatim}
summary BUSTRAVL FARE GASPRICE INCOME POP DENSITY LANDAREA --simple
corr BUSTRAVL FARE GASPRICE INCOME POP DENSITY LANDAREA --plot="display"  
fare_bustravl     <- gnuplot BUSTRAVL FARE     --output="display"
gasprice_bustravl <- gnuplot BUSTRAVL GASPRICE --output="display"
income_bustravl   <- gnuplot BUSTRAVL INCOME   --output="display"
pop_bustravl      <- gnuplot BUSTRAVL POP      --output="display"
density_bustravl  <- gnuplot BUSTRAVL DENSITY  --output="display"
landarea_bustravl <- gnuplot BUSTRAVL LANDAREA --output="display"
\end{verbatim}

\item[{Actividad 3}] Estime por MCO el modelo inicial completo y guárdelo como un icono
\begin{verbatim}
ModeloInicial <- ols BUSTRAVL 0 FARE GASPRICE INCOME POP DENSITY LANDAREA
modeltab add
\end{verbatim}
El nombre (en la línea anterior \texttt{ModeloInicial}) puede ser el que
usted elija, pero si tiene espacios en blanco debe escribirlo entre
comillas, por ejemplo \texttt{"Modelo uno"} \texttt{<- ols Y 0 x1 x2}). Recuerde
además que Gretl entiende la variable \texttt{0} como el regresor
constante).

Compare los resultados con sus expectativas iniciales.

\begin{itemize}
\item Observe qué variables son significativas y cuáles no. ¿Cuál es la
menos significativa?

\item Observe también el coeficiente de determinación ajustado, así como
los criterios de información para la selección de modelos
``Criterio de información de Akaike (AIC)'' y ``Criterio de
información Bayesiano de Schwarz (BIC)'' para compararlos con
otros alternativos más adelante.
\end{itemize}

\item[{Actividad 4}] La variable con mayor \emph{p}-valor (``la menos
significativa'') es el precio del combustible \texttt{GASPRICE}.
Realice una nueva regresión omitiendo esta variable.

Puede hacerlo partiendo del modelo anterior, o definiendo un nuevo
modelo (le recomiendo la primera):

\begin{itemize}
\item Partiendo del modelo anterior podemos omitir la variable:

\begin{itemize}
\item Desde la ventana del modelo estimado ``pinche'' en \textbf{\emph{\texttt{contrastes
      -{}-> omitir variables}}} seleccione \texttt{GASPRICE}; o bien
ejecute\footnote{el comando opuesto a \texttt{omit} es \texttt{add}}
\begin{verbatim}
omit GASPRICE
\end{verbatim}

\item Alternativamente podemos definir un nuevo modelo sin \texttt{GASPRICE}.
``pinchando'' en \textbf{\emph{\texttt{Modelo -{}-> Mínimos Cuadrados Ordinarios}}} y
seleccionando todos los regresores menos \texttt{GASPRICE}, o bien ejecutando
\texttt{ols BUSTRAVL 0 FARE INCOME POP DENSITY LANDAREA}
\end{itemize}

Analice los resultados:

\begin{itemize}
\item ¿Mejora la precisión de las estimaciones? ¿Qué variable es ahora
la menos significativa?

\item Observe también el coeficiente de determinación ajustado, así
como los criterios de información para la selección de modelos
``Criterio de información de Akaike (AIC)'' y ``Criterio de
información Bayesiano de Schwarz (BIC)''. ¿Le parece que ha
mejorado el modelo?
\end{itemize}
\end{itemize}

\item[{Actividad 5}] El mayor \emph{p}-valor corresponde ahora a \texttt{FARE}, pero
la teoría económica sugiere que el precio es una importante variable
a la hora de explicar la demanda de un bien o servicio. Por ello la
debemos mantener. La siguiente variable menos significativa (un
inaceptable \emph{p}-valor del 50\%) es la extensión de la ciudad
\texttt{LANDAREA}

Repita los pasos del punto anterior, pero omitiendo \texttt{LANDAREA}.
\begin{verbatim}
omit LANDAREA
\end{verbatim}

\item[{Actividad 6}] La variable \texttt{FARE} continua teniendo un \emph{p}-valor
excesivo (49\%). Esto sugiere, que dadas las demás variables
explicativas, la tarifa del billete no afecta demasiado a la demanda
de este servicio; es decir, los viajeros no parecer ser muy
sensibles al precio del billete\footnote{al menos para tarifas en torno a
las observadas en la muestra: alrededor de 0.8 dolares con una
tarifa mínima de 0.5\$ y una máxima de 1.5\$.}. Observe de nuevo el
diagrama de dispersión entre \texttt{FARE} y \texttt{BUSTRAVL} que siguiere este
comportamiento. Por tanto, veamos que pasa si finalmente omitimos la
variable \texttt{FARE} (si tenemos fundadas dudas sobre este comportamiento
de los consumidores, entonces \emph{no deberíamos eliminar la variable}
\texttt{FARE} del modelo).

Repita los pasos omitiendo \texttt{FARE}.
\begin{verbatim}
omit FARE 
\end{verbatim}

\item[{Actividad 7}] Las variables del modelo final son significativas y
los criterios de selección de modelos han mejorado, pero queda un
último paso.

Hemos ido ``quitando'' variables basándonos en los estadísticos
\Testadistico. Sabemos que esto es incorrecto puesto que
conjuntamente estas variables podrían ser significativas. Así pues,
debemos realizar un contraste de significación de las tres variables
conjuntamente; es decir, partiendo del modelo inicial debemos
contrastar
\begin{displaymath}
  \Hnula:\ \beta_2=0,\quad \beta_3=0,\quad \beta_7=0.
\end{displaymath}

\begin{itemize}
\item \emph{Partiendo del modelo inicial} omita de golpe las variables
\texttt{FARE}, \texttt{GASPRICE} y \texttt{LANDAREA} para obtener un modelo
restringido. Seleccione con el ratón la ventana del primer modelo
y en \textbf{\emph{\texttt{omitir}}} y marque las tres variables, o bien ejecute el
código:
\begin{verbatim}
ols BUSTRAVL 0 FARE GASPRICE INCOME POP DENSITY LANDAREA
ModeloFinal <- omit FARE GASPRICE LANDAREA
modeltab add
\end{verbatim}
¿Son conjuntamente significativas? ¿Mejora la precisión de las
estimaciones? ¿y los criterios de selección? ¿Son los signos de
los parámetros los esperados? A la luz de los resultados, viajar
en autobús ¿parece ser un bien normal, o un bien inferior?
\end{itemize}

\item[{Actividad 8}] Toda la inferencia que hemos realizado se basa en la
hipótesis de normalidad de los residuos. Contraste si se puede
rechazar o no la hipótesis de normalidad en ambos modelos (el
inicial y el final). El contraste Jarque-Bera es el más frecuente en
Econometría.
\begin{itemize}
\item En la ventana de resultados de un modelo ``pinche'' en
\textbf{\emph{\texttt{contrastes -{}-> Normalidad de los residuos}}}, o bien ejecute el
código:
\begin{verbatim}
series e = $uhat
normtest --all e
\end{verbatim}
Piense que los contrastes realizados en la anteriores actividades,
asumen la normalidad de las perturbaciones. Si esta hipótesis es
rechazada toda la inferencia de los contrates anteriores podría
quedar cuestionada.
\end{itemize}
\end{description}

\subsection{Código completo de la práctica \texttt{BusTravelers.inp}}
\label{sec:org29c0ce6}
\lstinputlisting{scripts/BusTravelers.inp}
\clearpage
\end{document}
% Created 2023-11-02 jue 20:44
% Intended LaTeX compiler: pdflatex
\documentclass[11pt]{article}
\usepackage[utf8]{inputenc}
\usepackage[T1]{fontenc}
\usepackage{graphicx}
\usepackage{longtable}
\usepackage{wrapfig}
\usepackage{rotating}
\usepackage[normalem]{ulem}
\usepackage{amsmath}
\usepackage{amssymb}
\usepackage{capt-of}
\usepackage{hyperref}
\usepackage[spanish, ]{babel}
\usepackage[margin=0.5in]{geometry}
\usepackage[svgnames,x11names]{xcolor}
\hypersetup{linktoc = all, colorlinks = true, urlcolor = DodgerBlue4, citecolor = PaleGreen1, linkcolor = SpringGreen4}
\PassOptionsToPackage{hyphens}{url}
\usepackage{nacal}
\usepackage{framed}
\usepackage{listings}
\input{hansl.tex}
\lstnewenvironment{hansl-gretl}
{\lstset{language={hansl},basicstyle={\ttfamily\footnotesize},numbers,rame=single,breaklines=true}}
{}
\newcommand{\hansl}[1]{\lstset{language={hansl},basicstyle={\ttfamily\small}}\lstinline{#1}}
\lstset{backgroundcolor=\color{lightgray!20}, }
\author{Marcos Bujosa}
\date{\today}
\title{Lección 3}
\begin{document}

\maketitle
\tableofcontents

\clearpage

\section{Datos de Anscombe}
\label{sec:orgcc3c40f}
\begin{center}
\begin{tabular}{ll}
Guión: & \href{https://github.com/mbujosab/Ectr/tree/master/Practicas/Gretl/scripts/anscombe.inp}{anscombe.inp}\\[0pt]
\end{tabular}
\end{center}

En esta práctica con \href{https://gretl.sourceforge.net/es.html}{Gretl} trabajaremos con los \href{http://en.wikipedia.org/wiki/Anscombe's\_quartet}{datos} diseñados por
F.J. Anscombe para ilustrar la importancia de \emph{ver} los diagramas de
dispersión entre distintas variables antes de realizar ninguna
regresión y así identificar deficiencias en el planteamiento de los
modelo, pues si únicamente se analizan los resultados numéricos de las
estimaciones dichas deficiencias quedan ocultas.

\subsection{La importancia de \emph{ver} los datos}
\label{sec:orgca59e93}
\begin{enumerate}
\item Objetivo
\label{sec:orgfde63c1}
Cuando se ajusta un modelo a los datos, es NECESARIO comenzar
observando gráficamente los datos. F.J. Anscombe diseñó este
\href{http://en.wikipedia.org/wiki/Anscombe's\_quartet}{conjunto de datos}
para ilustrar la importancia de representar gráficamente los datos
antes de realizar un análisis empírico.
\end{enumerate}

\subsection{Los datos}
\label{sec:org4d4cad3}

\begin{center}
\begin{tabular}{rrrrrr}
\Vect[1]{y} & \Vect[2]{y} & \Vect[3]{y} & \Vect[4]{y} & \Vect{x} & \Vect[4]{x}\\[0pt]
\hline
8.04 & 9.14 & 7.46 & 6.58 & 10 & 8\\[0pt]
6.95 & 8.14 & 6.77 & 5.76 & 8 & 8\\[0pt]
7.58 & 8.74 & 12.74 & 7.71 & 13 & 8\\[0pt]
8.81 & 8.77 & 7.11 & 8.84 & 9 & 8\\[0pt]
8.33 & 9.26 & 7.81 & 8.47 & 11 & 8\\[0pt]
9.96 & 8.10 & 8.84 & 7.04 & 14 & 8\\[0pt]
7.24 & 6.13 & 6.08 & 5.25 & 6 & 8\\[0pt]
4.26 & 3.10 & 5.39 & 12.50 & 4 & 19\\[0pt]
10.84 & 9.13 & 8.15 & 5.56 & 12 & 8\\[0pt]
4.82 & 7.26 & 6.42 & 7.91 & 7 & 8\\[0pt]
5.68 & 4.74 & 5.73 & 6.89 & 5 & 8\\[0pt]
\hline
\end{tabular}
\end{center}

\subsection{Los estadísticos de los datos}
\label{sec:org9e38432}


\begin{center}
\begin{tabular}{lr}
\hline
Estadísticos de los datos & \\[0pt]
\hline
Media de cada una de las variables \Vect{x}: & 9.0\\[0pt]
Varianza de cada una de las variables \Vect{x}: & 11.0\\[0pt]
Media de cada una de las variables \Vect{y}: & 7.5\\[0pt]
Varianza de cada una de las variables \Vect{y}: & 4.12\\[0pt]
Correlación entre las variables \Vect{x} e \Vect{y} de cada regresión: & 0.816\\[0pt]
\hline
\end{tabular}
\end{center}

\begin{center}
\begin{tabular}{lr}
\hline
Rectas de regresión & R\textsuperscript{2}\\[0pt]
\hline
\(\VEstmc[1]{y} = 3\cdot\Vect{1} + 0.5\cdot\Vect{x}\) & 0.67\\[0pt]
\(\VEstmc[2]{y} = 3\cdot\Vect{1} + 0.5\cdot\Vect{x}\) & 0.67\\[0pt]
\(\VEstmc[3]{y} = 3\cdot\Vect{1} + 0.5\cdot\Vect{x}\) & 0.67\\[0pt]
\(\VEstmc[4]{y} = 3\cdot\Vect{1} + 0.5\cdot\Vect[4]{x}\) & 0.67\\[0pt]
\hline
\end{tabular}
\end{center}

\subsection{Actividad 1 - Estadísticos descriptivos}
\label{sec:org6b87beb}

\begin{enumerate}
\item Carga de datos
\label{sec:org4f8bde7}
\textbf{\emph{\texttt{Archivo -{}-> Abrir datos -{}-> Archivo de muestra}}} y en la pestaña
\textbf{\emph{\texttt{Gretl}}} seleccione \texttt{anscombe}.

{\vspace{3pt} \color{gray!70!black}
\emph{o bien teclee en linea de comandos}:
\begin{verbatim}
open anscombe
\end{verbatim}
}

\item Visualice los estadísticos descriptivos de las datos
\label{sec:orgfa5bd91}

\begin{itemize}
\item Marque una variable (o varias) y ``Pinche'' con el botón derecho.
Elija \textbf{\emph{\texttt{Estadísticos principales}}}

{\vspace{3pt} \color{gray!70!black}
\emph{o bien teclee en linea de comandos} \texttt{summary} seguido de las series. Por ejemplo:
\begin{verbatim}
summary --simple y1 y2 y3 y4 x x4
\end{verbatim}
}
\end{itemize}

\item Observar la correlación entre variables
\label{sec:org1149a6f}

\begin{itemize}
\item \textbf{\emph{\texttt{Ver -{}-> Matriz de correlación}}} y elija variables

{\vspace{3pt} \color{gray!70!black}
\emph{o, por ejemplo, teclee en linea de comandos}:
\begin{verbatim}
corr y1 y2 y3 y4 x x4
\end{verbatim}
}
\end{itemize}
\end{enumerate}

\subsection{Actividad 2 - Cuatro regresiones}
\label{sec:orgd632461}

Realice los siguientes cuatro ajustes MCO
\begin{itemize}
\item Modelo 1: \(\quad \VEstmc[1]{y} = \Estmc{a}\Vect{1} + \Estmc{b}\Vect{x}\)
\item Modelo 2: \(\quad \VEstmc[2]{y} = \Estmc{a}\Vect{1} + \Estmc{b}\Vect{x}\)
\item Modelo 3: \(\quad \VEstmc[3]{y} = \Estmc{a}\Vect{1} + \Estmc{b}\Vect{x}\)
\item Modelo 4: \(\quad \VEstmc[4]{y} = \Estmc{a}\Vect{1} + \Estmc{b}\Vect[4]{x}\)
\end{itemize}

y compare los resultados.

\begin{itemize}
\item Para cada modelo:

\textbf{\emph{\texttt{Modelo -> Mínimos Cuadrados Ordinarios}}}; indique regresando y
regresores. Pulse \textbf{\emph{\texttt{Aceptar}}} (guarde el modelo como icono).

{\vspace{3pt} \color{gray!70!black}
\emph{o bien teclee en linea de comandos}:
\begin{verbatim}
Modelo1 <- ols y1 0 x
Modelo2 <- ols y2 0 x
Modelo3 <- ols y3 0 x
Modelo4 <- ols y4 0 x4
\end{verbatim}
}
\end{itemize}

\subsection{Actividad 3 - Discusión}
\label{sec:org36f7834}

\begin{enumerate}
\item Compare los estadísticos de los distintos modelos
\label{sec:org55b87c1}
\begin{itemize}
\item el coeficiente de determinación y en el coeficiente de determinación ajustado
\item la suma de cuadrados de los residuos,
\item la desviación típica de los errores de ajuste (\texttt{D.T. de la regresión}),
\item los estadísticos \testadistico
\item los p-valores.
\end{itemize}

\item A la luz de los estadísticos de las cuatro regresiones ¿Qué modelo es mejor?
\label{sec:org62f084b}

\item Observe el diagrama de dispersión XY en cada modelo
\label{sec:org86a7b08}

\begin{itemize}
\item ``pinche'' en \textbf{\emph{\texttt{Ver -{}-> Gráficos -{}-> Gráfico XY (scatter)}}}
Elija la variable para el eje X (regresor) y la variable Y
(regresando)
\end{itemize}

{\vspace{3pt} \color{gray!70!black}
\emph{o, por ejemplo, teclee en linea de comandos}:
\begin{verbatim}
gnuplot y1 x
\end{verbatim}
\emph{podemos pintar varios diagramas juntos con}:
\begin{verbatim}
gnuplot y1 y2 y3 x
\end{verbatim}
\emph{o varios diagramas separados con}:
\begin{verbatim}
scatters y1 y2 y3 ; x
scatters y4 ; x4
\end{verbatim}
}

\vspace{-10pt}

\item De los cuatro modelos\dots{} ¿cuáles parecen razonables?
\label{sec:orgabdf6e1}


\vspace{16pt}
\noindent
\textbf{Código completo de la práctica} \texttt{anscombe.inp}
\vspace{10pt}
\lstinputlisting{scripts/anscombe.inp}
\clearpage
\end{enumerate}

\section{Los errores de ajuste MCO son perpendiculares a los regresores}
\label{sec:org891e0ed}
\begin{center}
\begin{tabular}{ll}
Guión: & \href{https://github.com/mbujosab/Ectr/tree/master/Practicas/Gretl/scripts/TextilTheil.inp}{TextilTheil.inp}\\[0pt]
\end{tabular}
\end{center}

\begin{enumerate}
\item Carga de datos
\label{sec:org0cca9e9}

Vamos a usar el conjunto de datos de consumo percápita de textiles, de
Henri Theil, Principios de Econometría, Nueva York: Wiley, 1971,
p. 102.  El conjunto de datos consta de 17 observaciones anuales de
series de tiempo para el periodo 1923--1939 del consumo de textiles en
los Países Bajos. Todas las variables son expresadas como índices con
base 100 en 1925.

\textbf{\emph{\texttt{Archivo -{}-> Abrir datos -{}-> Archivo de muestra}}} y en la pestaña
\textbf{\emph{\texttt{Gretl}}} seleccione \texttt{theil}.

{\vspace{3pt} \color{gray!70!black}
\emph{o bien teclee en linea de comandos}:
\begin{verbatim}
open theil
\end{verbatim}
}

\item Ajuste por MCO el modelo
\label{sec:org1b863df}
\begin{displaymath}
  \Vect{y}=\Estmc{\beta_1}\Vect{1}+\Estmc{\beta_2}\Vect[2]{X}+\Estmc{\beta_3}\Vect[3]{X}+\res
\end{displaymath}
donde \Vect{y} es el consumo de textiles \Vect[2]{X} la renta y
\Vect[3]{X} los precios relativos: \textbf{\emph{\texttt{Modelo -> Mínimos Cuadrados
Ordinarios}}}; indique \texttt{consume} como regresando y \texttt{const}, \texttt{income} y
\texttt{relprice} como regresores. Pulse \textbf{\emph{\texttt{Aceptar}}} (guarde el modelo como
icono).

{\vspace{3pt} \color{gray!70!black}
\emph{o bien teclee en linea de comandos}:
\begin{verbatim}
AjusteMCO <- ols consume const income relprice
\end{verbatim}
}

\item Guardado de datos ajustados y de los errores
\label{sec:org971db00}

En la ventana del modelo ajustado: \textbf{\emph{\texttt{Guardar -> Valores estimados}}}
e indique un nombre, por ejemplo \texttt{yhat}. Lo mismo para los errores:
\textbf{\emph{\texttt{Guardar -> Residuos}}} y como nombre por ejemplo \texttt{ehat}

{\vspace{3pt} \color{gray!70!black}
\emph{o bien teclee en linea de comandos}:
\begin{verbatim}
series ehat = $uhat
series yhat = $yhat
\end{verbatim}
}

\item Verificación de que los residuos son ortogonales a los regresores y al ajuste, pero no al regresando
\label{sec:org03fd488}

Compruebe que
\begin{displaymath}
  \media{\res}=0,\quad
  \mediap*{\prodH{\res}{\Vect[2]{x}}}=0,\quad
  \mediap*{\prodH{\res}{\Vect[3]{x}}}=0,\quad
  \mediap*{\prodH{\res}{\VEstmc{y}}} =0\quad\text{pero}\quad
  \mediap*{\prodH{\res}{\Vect{y}}}\ne0.
\end{displaymath}

\begin{itemize}
\item En la ventana principal: \textbf{\emph{\texttt{Añadir -> Definir nueva variable}}} y en
cada caso escribir la formula y pulsar en \textbf{\emph{\texttt{Aceptar}}}
\begin{enumerate}
\item ei = ehat*income
\item er = ehat*relprice
\item ey = ehat*yhat
\item ec = ehat*consume
\end{enumerate}
\end{itemize}

{\vspace{3pt} \color{gray!70!black}
\emph{o bien teclee en linea de comandos}:
\begin{verbatim}
series ei = ehat*income
series er = ehat*relprice
series ey = ehat*yhat
series ec = ehat*consume
\end{verbatim}
}

\begin{itemize}
\item Observe los valores medios de los productos, es decir, de \texttt{ehat},
\texttt{ei}, \texttt{er}, \texttt{ey} y \texttt{ec} marcando las variables haciendo click sobre
ellas con el botón derecho y eligiendo \textbf{\emph{\texttt{Estadísticos
  principales}}}
\end{itemize}

{\vspace{3pt} \color{gray!70!black}
\emph{o bien teclee en linea de comandos}:
\begin{verbatim}
summary --simple ehat ei er ey ec
\end{verbatim}
}

\item Explique los resultados.
\label{sec:org55fdf3e}


\vspace{16pt}
\noindent
\textbf{Código completo de la práctica} \texttt{TextilTheil.inp}
\vspace{10pt}
\lstinputlisting{scripts/TextilTheil.inp}
\clearpage
\end{enumerate}


\section{El coeficiente de determinación como cuadrado de la correlación entre valores observados y ajustados}
\label{sec:orge6215fa}
\begin{center}
\begin{tabular}{ll}
Guión: & \href{https://github.com/mbujosab/Ectr/tree/master/Practicas/Gretl/scripts/EjPviviendaR2.inp}{EjPviviendaR2.inp}\\[0pt]
\end{tabular}
\end{center}

Calcule el coeficiente de determinación \(R^2\) para el ejemplo del
precio de las viviendas, pero empleando el coeficiente de correlación
entre los precios y los precios ajustados.  (\textbf{Pista:} calcule
el coeficiente de correlación lineal simple entre \VEstmc{Y} y
\Vect{Y} y elévelo al cuadrado.) Puede hacerlo mediante los menús
desplegables de las ventanas de Gretl, \emph{o bien en linea de comandos}:

\vspace{16pt}
{\vspace{3pt} \color{gray!70!black}
}

\lstinputlisting{scripts/EjPviviendaR2.inp}

\clearpage


\section{La importancia a los criterios de ajuste es muy relativa}
\label{sec:orge2f9a95}
\begin{center}
\begin{tabular}{ll}
Guión: & \href{https://github.com/mbujosab/Ectr/tree/master/Practicas/Gretl/scripts/PesoEdad.inp}{PesoEdad.inp}\\[0pt]
\end{tabular}
\end{center}

\begin{itemize}
\item Cargue los datos del ejemplo del peso y edad de ocho niños.

\begin{itemize}
\item Puede descargar el fichero \texttt{PesoEdad.gdt} del subdirectorio
\texttt{datos} del directorio con el material del curso,

\item o introducir los datos manualmente siguiendo \textbf{\emph{\texttt{Archivo ->
       Nuevo conjunto de datos}}}. Indique que hay 8 observaciones de
sección cruzada, y marque ``empezar a introducir los valores de
los datos''. Introduzca el nombre de la primera variable y
luego los datos del peso de cada niño. Pulsando en \texttt{+} puede
añadir la segunda variable.
\end{itemize}

\item Genere la serie de edades al cuadrado y de la de edades al cubo.

\item Ajuste el modelo
\(\VEstmc{peso}=\Estmc{\beta_1}\Vect{1}+\Estmc{\beta_2}\Vect{edad}\)
y añádalo a la tabla de modelos.
\end{itemize}


\begin{itemize}
\item Guarde el modelo como icono y pulse sobre su icono con el
botón derecho. Seleccione \textbf{\emph{\texttt{Añadir a la tabla de modelos}}}

\begin{itemize}
\item o bien, tras estimar el modelo teclee \texttt{modeltab add}
\end{itemize}
\end{itemize}
\begin{itemize}
\item Ajuste
\(\VEstmc{peso}=\Estmc{\beta_1}\Vect{1}+\Estmc{\beta_2}\Vect{edad}+\Estmc{\beta_3}\Vect[][2]{(edad)}\)
y añádalo a la tabla de modelos.

\item Ajuste
\(\VEstmc{peso}=\Estmc{\beta_1}\Vect{1}+\Estmc{\beta_2}\Vect{edad}+\Estmc{\beta_3}\Vect[][2]{(edad)}+\Estmc{\beta_3}\Vect[][3]{(edad)}\)
y añádalo a la tabla de modelos.

\item Compare los ajustes: pinchando sobre el icono de \texttt{Tabla de
     modelos}; o bien tecleando \texttt{modeltab show}.

\item Genere las figuras de los distintos ajustes.
\end{itemize}

Fíjese que que en los dos últimos ajustes las potencias de la edad son
regresores, pero que en los tres modelos la única variable
\emph{explicativa} del peso es la edad (variable explicativa y regresor no
son sinónimos).

El siguiente guión realiza todos los pasos
{\vspace{3pt} \color{gray!70!black}
}

\vspace{12pt}
\lstinputlisting{scripts/PesoEdad.inp}


\clearpage


\section{¿Tiene sentido llamar variable explicativa a cualquier regresor?}
\label{sec:org7980e6d}
\begin{center}
\begin{tabular}{ll}
Guión: & \href{https://github.com/mbujosab/Ectr/tree/master/Practicas/Gretl/scripts/cigfecfr.inp}{cigfecfr.inp}\\[0pt]
\end{tabular}
\end{center}

\emph{\textbf{Una regresión infantil:} exploremos la teoría que ``Dumbo''
   ofrece a los niños sobre la natalidad y las cigüeñas.}

¿Podemos encontrar relación entre la tasa de fecundidad de las
mujeres francesas (\texttt{fec}) y la densidad de cigüeñas (\texttt{cig}) en
Alsacia para el período 1945-1986 (\texttt{annee})? La tasa de fecundidad
está calculada como número de niños por 10000 mujeres (Indicateur
conjucturel de fécondité en 2004 par l'INSEE \url{http://www.insee.fr}).
Las cifras de cigüeñas proceden de The Global Population Database:
NERC Centre for Population Biology
(\url{http://www3.imperial.ac.uk/cpb/research/patternsandprocesses/gpdd})
y se trata del número de parejas de cigüeñas que anidan en la
región de Alsacia.

\begin{itemize}
\item Cargue el conjunto de datos \texttt{cigfecfr.inp}.

\item Realice un diagrama de dispersión entre \texttt{fec} y \texttt{cig} y calcule
el coeficiente de correlación.

\item Ajuste por MCO la tasa de fecundidad \texttt{fec} con la constante y \texttt{cig}

\item Realice un gráfico de series temporales de ambas
variables. Observe que parece haber un retardo entre la aparición
de las cigüeñas y la variación en la tasa de natalidad.

\item Cree una nueva serie \texttt{cig6} que sea la serie \texttt{cig} retardada 6
meses y repita los pasos anteriores. Observe que el ajuste
mejora.

\item A la luz de los resultados ¿``explica'' el número de parejas de
cigüeñas casi el 90\% de la variabilidad en la natalidad de la
región de Alsacia en esos años?
\end{itemize}

El siguiente guión realiza todos los pasos
{\vspace{3pt} \color{gray!70!black}
}

\vspace{12pt}
\lstinputlisting{scripts/cigfecfr.inp}
\end{document}
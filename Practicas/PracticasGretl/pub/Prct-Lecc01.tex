% Created 2024-07-26 vie 10:02
% Intended LaTeX compiler: pdflatex
\documentclass[11pt]{article}
\usepackage[utf8]{inputenc}
\usepackage[T1]{fontenc}
\usepackage{graphicx}
\usepackage{longtable}
\usepackage{wrapfig}
\usepackage{rotating}
\usepackage[normalem]{ulem}
\usepackage{amsmath}
\usepackage{amssymb}
\usepackage{capt-of}
\usepackage{hyperref}
\usepackage[spanish, ]{babel}
\usepackage[margin=0.5in]{geometry}
\usepackage[svgnames,x11names]{xcolor}
\hypersetup{linktoc = all, colorlinks = true, urlcolor = DodgerBlue4, citecolor = PaleGreen1, linkcolor = SpringGreen4}
\PassOptionsToPackage{hyphens}{url}
\usepackage{nacal}
\usepackage{framed}
\usepackage{listings}
\lstset{language = hansl, %
        showstringspaces = false, % 
        basicstyle = \footnotesize\ttfamily,
        keywordstyle = \color{red!50!black}\ttfamily,
        keywordstyle = [2]\color{green!40!black}\bfseries,
        keywordstyle = [3]\color{cyan!65!black}\ttfamily,
        keywordstyle = [4]\color{green!55!black}\ttfamily,
        commentstyle = \color{blue!70!black}\sffamily,       % 
        stringstyle = \color{magenta!75!black}\ttfamily,
        % backgroundcolor=\color{lightgray},
        % These commands partly sorted by meaning:
        morekeywords = {list, bundle, %
                matrices, strings, bundles, lists, %
                funcerr, hfplot, launch, midasreg, %
                pkg, plot, setopt, % 
                deriv, params, %
                for, foreach, while, 
                },
        morekeywords = [2]{array, defarray, null, %
                BFGSmin, BFGScmax, BFGScmin, %
                bread, bwrite, cdummify, cnumber, %
                cnameget, cnameset, dropcoll, %
                easterday, ecdf, exists, fevd, genseries, %
                getinfo, getkeys, GSSmax, GSSmin, %
                hfdiff, hfldiff, hflags, hflist, instring, %
                isdiscrete, isdummy, isoconv, isodate, %
                jsonget, jsongetb, juldate, kdsmooth, %
                kmeier, kpsscrit, ksetup, linearize, %
                lrcovar, mgradient, mlincomb, mweights, %
                naalen, nlines, NMmax, NMmin, %
                normtest, npcorr, NRmin, numhess, %
                pexpand, pxnobs, qlrpval, %             
                rnameget, rnameset, %
                seasonals, sleep, smplspan, sprintf, %
                square, stringify, strvals, substr, %
                typeof, varnames, xmlget, series},
        % the following for the double-dash options:
        %% moredelim       = [l][\color{green!50!gray}\ttfamily]{--},
        % obsolete former functions:
        deletekeywords = [2]{isnull, rowname, rownames, %
                colname, colnames, xpx },
        % and obsolete former commands:
        deletekeywords = {kalman, shell, sscanf},
        % try to redefine / override the $-accessors (not as group-1-keywords)
        keywordsprefix = {\$\$}, % anything which isn't a single $
        morekeywords = [3]{\$coeff,\$uhat, % to be completed
                \$ahat, \$aic, \$bic, \$chisq,
                \$command, \$compan, \$datatype, \$depvar, \$df,
                \$diagpval, \$diagtest, \$dw, \$dwpval, \$ec,
                \$error, \$ess , \$evals, \$fcast, \$fcse,
                \$fevd, \$Fstat, \$gmmcrit, \$h, \$hausman,
                \$hqc, \$huge, \$jalpha, \$jbeta, \$jvbeta,
                \$lang, \$llt, \$lnl, \$macheps, \$mnlprobs,
                \$model, \$ncoeff, \$nobs, \$nvars, \$obsdate,
                \$obsmajor, \$obsmicro, \$obsminor, \$pd , \$pi,
                \$pvalue, \$qlrbreak, \$rho, \$rsq, \$sample,
                \$sargan, \$sigma, \$stderr, \$stopwatch, \$sysA,
                \$sysB, \$sysGamma, \$sysinfo, \$system, \$T,
                \$t1, \$t2, \$test, \$tmax, \$trsq,
                \$uhat, \$unit, \$vcv, \$vecGamma, \$version,
                \$vma, \$windows, \$xlist, \$xtxinv, \$yhat,
                \$ylist},
        morekeywords = [4]{scalar}, %
        %morekeywords = [5]{randgen,mean,std}, %
    inputencoding = utf8,  % Input encoding
    extendedchars = true,  % Extended ASCII
    texcl         = true,  % Activate LaTeX commands in comments
    literate      =        % Support additional characters
      {á}{{\'a}}1  {é}{{\'e}}1  {í}{{\'i}}1 {ó}{{\'o}}1  {ú}{{\'u}}1
      {Á}{{\'A}}1  {É}{{\'E}}1  {Í}{{\'I}}1 {Ó}{{\'O}}1  {Ú}{{\'U}}1
      {ü}{{\"u}}1  {Ü}{{\"U}}1  {ñ}{{\~n}}1 {Ñ}{{\~N}}1  
}

\lstnewenvironment{hansl-gretl}
{\lstset{language={hansl},basicstyle={\ttfamily\footnotesize},numbers,rame=single,breaklines=true}}
{}
\newcommand{\hansl}[1]{\lstset{language={hansl},basicstyle={\ttfamily\small}}\lstinline{#1}}
\lstset{backgroundcolor=\color{lightgray!20}, }
\author{Marcos Bujosa}
\date{\today}
\title{Lección 1}
\begin{document}

\maketitle
\tableofcontents



\section{Número de viajeros internacionales}
\label{sec:orgcb2a902}
\begin{center}
\begin{tabular}{ll}
Guión: & \href{https://github.com/mbujosab/Ectr/tree/master/Practicas/Gretl/scripts/airlinePass.inp}{airlinePass.inp}\\[0pt]
\end{tabular}
\end{center}

En esta primera práctica con \href{https://gretl.sourceforge.net/es.html}{Gretl} reproduciremos el ejemplo visto en
clase, en el que hemos aplicado distintas transformaciones a los datos
hasta obtener una serie temporal que podemos asumir que podría ser una
realización de un proceso estocástico estacionario.  Los datos son
mensuales y contienen el número de viajeros internacionales (en miles)
en las líneas aéreas norteamericanas en los años 1949--1960 que
aparece en manual de Box \& Jenkins.

\begin{enumerate}
\item Objetivo
\label{sec:orgc922707}
\begin{enumerate}
\item Reproducir el primer ejemplo visto en clase.
\item Mostrar datos.
\item Transformalos
\item Generar gráficos.
\end{enumerate}

\item Carga de datos
\label{sec:orgc91ef83}
\textbf{\emph{\texttt{Archivo -{}-{}> Abrir datos -{}-{}> Archivo de muestra}}} y en la pestaña
\textbf{\emph{\texttt{Gretl}}} seleccione \texttt{bjg}.

{\vspace{0pt} \footnotesize \color{gray!70!black}
\emph{o bien teclee en linea de comandos}:
\begin{verbatim}
open bjg
\end{verbatim}
}
\end{enumerate}

\subsection{Actividad 1 - mostrar datos}
\label{sec:org47ba6de}
\begin{enumerate}
\item Visualice los datos de precios y tamaños de las casas
\label{sec:org831979d}
\begin{itemize}
\item En la ventana principal de \href{https://gretl.sourceforge.net/es.html}{Gretl}, marque con el ratón la 
variable: \texttt{g}.
\item ``Pinche'' sobre ella con el botón derecho del ratón.
\item Seleccione \textbf{\emph{\texttt{mostrar valores}}} del menú desplegable que se ha
abierto al pinchar.

{\vspace{1pt} \footnotesize \color{gray!70!black}
\emph{o bien teclee en linea de comandos}:
\begin{verbatim}
print -o g
\end{verbatim}
}
\end{itemize}

\vspace{-3pt}

\item Ayuda
\label{sec:orgc9b8d40}
Para consultar la documentación sobre cualquier comando, puede emplear
el menú desplegable \textbf{\emph{\texttt{Ayuda}}} que aparece arriba, a la derecha de la
ventana principal de \href{https://gretl.sourceforge.net/es.html}{Gretl}.
\begin{itemize}
\item \textbf{\emph{\texttt{Ayuda -> Guía de Instrucciones}}} y ``pinche'' sobre \textbf{\emph{\texttt{print}}}
\end{itemize}

{\vspace{0pt} \footnotesize \color{gray!70!black}
\emph{o bien teclee en linea de comandos}: \texttt{help print}
}
\end{enumerate}

\subsection{Actividad 2 - Gráfico de series temporales}
\label{sec:org2b31b9a}
\begin{enumerate}
\item Scatter plot
\label{sec:org0aa7ca4}
\begin{itemize}
\item Marque la variable \texttt{g} (pulsando \texttt{ctrl} y pinchando con el botón
derecho del ratón sobre ella). Elija \textbf{\emph{\texttt{Gráfico de series temporales}}}

{\vspace{1pt} \footnotesize \color{gray!70!black}
\emph{o bien teclee en linea de comandos}: \texttt{gnuplot g -{}-{}time-series -{}-{}with-lines}
}
\end{itemize}

\item Guardar gráfico como \emph{icono} para editarlo más tarde
\label{sec:orgf3eeb66}
\begin{itemize}
\item ``Pinche'' con el botón derecho sobre la ventana del gráfico.
\item Seleccione \textbf{\emph{\texttt{Guardar a sesión como icono}}}

{\vspace{1pt} \footnotesize \color{gray!70!black} \color{gray!70!black}
\emph{o bien teclee en linea de comandos}:
\begin{verbatim}
AirlinePassengers <- gnuplot g --time-series --with-lines
\end{verbatim}
(\texttt{AirlinePassengers} \emph{es el nombre con el que se guardará el icono})
}
\end{itemize}

En la zona inferior izquierda de la ventana principal puede ver una
serie de iconos. Uno de ellos es la \textbf{\emph{\texttt{vista de iconos de sesión}}}.
\end{enumerate}


\subsection{Actividad 3 - Transformar logarítmicamente los datos}
\label{sec:org726c40f}
Aunque el fichero ya contiene el logaritmo de la serie, vamos a
tranfformar logarítmicamente los datos originales.

Selecione con el ratón la variable \texttt{g} y luego pulse en el menú desplegable \textbf{\emph{\texttt{Añadir}}} que aparece arriba, en el centro de la
ventana principal de \href{https://gretl.sourceforge.net/es.html}{Gretl}.
\begin{itemize}
\item \textbf{\emph{\texttt{Añadir -> Logaritmos de las variables seleccionadas}}}

{\vspace{0pt} \footnotesize \color{gray!70!black}
\emph{o bien teclee en linea de comandos}: 
\begin{verbatim}
logs g
\end{verbatim}
}
\end{itemize}

Entre las variables aparecerá una nueva con el prefijo \texttt{l\_}, es decir,
en este caso aparecerá la variable \texttt{l\_g} (que contiene exactamente la
misma serie temporal que \texttt{lg}).

Genere el gráfico de series temporales de esta nueva serie y guardelo
como un nuevo icono (use un nombre suficientemente descriptivo para el
icono, por ejemplo \texttt{LogsAirlinePassengers})

{\vspace{0pt} \footnotesize \color{gray!70!black}
}

\subsection{Actividad 4 - Primera diferencia de los datos en logaritmos}
\label{sec:orgeefed41}

Selecione con el ratón la variable \texttt{l\_g} y luego pulse en el menú desplegable \textbf{\emph{\texttt{Añadir}}} que aparece arriba, en el centro de la
ventana principal de \href{https://gretl.sourceforge.net/es.html}{Gretl}.
\begin{itemize}
\item \textbf{\emph{\texttt{Añadir -> Primeras diferencias de las variables seleccionadas}}}

{\vspace{0pt} \footnotesize \color{gray!70!black}
\emph{o bien teclee en linea de comandos}: 
\begin{verbatim}
diff l_g
\end{verbatim}
}
\end{itemize}

Entre las variables aparecerá una nueva con el prefijo \texttt{d\_}, es decir,
en este caso aparecerá la variable \texttt{d\_l\_g}.

Genere el gráfico de series temporales de esta nueva serie y guardelo
como un nuevo icono (Use un nombre suficientemente descriptivo, por
ejemplo \texttt{D\_LogsAirlinePassengers})

{\vspace{0pt} \footnotesize \color{gray!70!black}
}

\subsection{Actividad 5 - El logaritmo no es una función lineal}
\label{sec:org52b496d}

Aunque el operador primera diferencia es lineal, la función logaritmo
no lo es. Comprobemos que no es lo mismo la primera diferencia del
logaritmo (calculado en la actividad anterior) que el logaritmo de la
diferencia.

\begin{itemize}
\item Añada la primera diferencia de \texttt{g} y luego el logaritmo de \texttt{d\_g}.
\item Marque con el ratón \texttt{d\_l\_g} y \texttt{l\_d\_g} y muestre sus valores; verá
que son distintos (no solo eso, dado que la función logaritmo solo
está definida para números positivos, en \texttt{l\_d\_g} parecen una gran
cantidad de valores ausentes).

{\vspace{0pt} \footnotesize \color{gray!70!black}
\emph{en linea de comandos}: 
\begin{verbatim}
diff g
logs d_g
print -o d_l_g l_d_g
\end{verbatim}
}
\end{itemize}

\subsection{Actividad 6 - Diferencia de orden 12 (o estacional) de la primera diferencia de los datos en logaritmos}
\label{sec:org831b7a5}
Selecione con el ratón la variable \texttt{d\_l\_g} y luego pulse en el menú desplegable \textbf{\emph{\texttt{Añadir}}} que aparece arriba, en el centro de la
ventana principal de \href{https://gretl.sourceforge.net/es.html}{Gretl}.
\begin{itemize}
\item \textbf{\emph{\texttt{Añadir -> Diferencias estacionales de las variables seleccionadas}}}

{\vspace{0pt} \footnotesize \color{gray!70!black}
\emph{o bien teclee en linea de comandos}: 
\begin{verbatim}
sdiff d_l_g
\end{verbatim}
}
\end{itemize}

Entre las variables aparecerá una nueva con el prefijo \texttt{sd\_}, es
decir, en este caso aparecerá la variable \texttt{sd\_d\_l\_g}.

Genere el gráfico de series temporales de esta nueva serie y guardelo
como un nuevo icono (Use un nombre suficientemente descriptivo, por
ejemplo \texttt{D12\_D\_LogsAirlinePassengers})

{\vspace{0pt} \footnotesize \color{gray!70!black}
}

Observe que en la serie obtenida ya no se observa ni tendencia ni un
componente cíclico estacional.

\subsection{Actividad 5 - El orden en que se aplican los operadores diferencia y diferencia estacional es irelevante}
\label{sec:org7365299}

\begin{enumerate}
\item calcule la diferencia estacional de la serie en logaritmos \texttt{l\_g} y
genere su gráfico

{\vspace{0pt} \footnotesize \color{gray!70!black}
\emph{en linea de comandos}: 
}

Observe que en la serie obtenida ya no el componente cíclico
estacional, pero sin embargo el promedio de cada año "deambula"
alrededor del valor \(0.1\) en ciclos de unos 4 años

\item ahora tome una primera diferencia de la serie anterior (sd\textsubscript{l}\textsubscript{g}) y
compruebe las diferencias entre la serie resultante (\texttt{d\_sd\_l\_g}) y
la obtenida en la actividad anterior (\texttt{sd\_d\_l\_g}).

{\vspace{0pt} \footnotesize \color{gray!70!black}
\emph{en linea de comandos}: 
}
\end{enumerate}

Es decir, el orden en que se tomen la diferencia ordinaria y la
diferencia estacional es irrelevante (pero recuerde que no pasa lo
mismo con la transformación logarítmica, que debe la primera
transformación aplicada a los datos).

\vspace{10pt}
\noindent
\textbf{Código completo de la práctica} \texttt{airlinePass.inp}
\vspace{10pt}
\lstinputlisting{scripts/airlinePass.inp}
\clearpage
\end{document}

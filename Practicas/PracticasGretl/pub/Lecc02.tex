% Created 2023-11-02 jue 20:44
% Intended LaTeX compiler: pdflatex
\documentclass[11pt]{article}
\usepackage[utf8]{inputenc}
\usepackage[T1]{fontenc}
\usepackage{graphicx}
\usepackage{longtable}
\usepackage{wrapfig}
\usepackage{rotating}
\usepackage[normalem]{ulem}
\usepackage{amsmath}
\usepackage{amssymb}
\usepackage{capt-of}
\usepackage{hyperref}
\usepackage[spanish, ]{babel}
\usepackage[margin=0.5in]{geometry}
\usepackage[svgnames,x11names]{xcolor}
\hypersetup{linktoc = all, colorlinks = true, urlcolor = DodgerBlue4, citecolor = PaleGreen1, linkcolor = SpringGreen4}
\PassOptionsToPackage{hyphens}{url}
\usepackage{nacal}
\usepackage{framed}
\usepackage{listings}
\input{hansl.tex}
\lstnewenvironment{hansl-gretl}
{\lstset{language={hansl},basicstyle={\ttfamily\footnotesize},numbers,rame=single,breaklines=true}}
{}
\newcommand{\hansl}[1]{\lstset{language={hansl},basicstyle={\ttfamily\small}}\lstinline{#1}}
\lstset{backgroundcolor=\color{lightgray!20}, }
\author{Marcos Bujosa}
\date{\today}
\title{Lección 2}
\begin{document}

\maketitle
\tableofcontents

\clearpage
\section{Precio de casas unifamiliares}
\label{sec:org0eafe95}
\begin{center}
\begin{tabular}{ll}
Guión: & \href{https://github.com/mbujosab/Ectr/tree/master/Practicas/Gretl/scripts/EjPvivienda.inp}{EjPvivienda.inp}\\[0pt]
\end{tabular}
\end{center}

En esta primera práctica con \href{https://gretl.sourceforge.net/es.html}{Gretl} reproduciremos el ejemplo visto
repetidamente en clase. Los datos corresponden a los precios de venta y
superficie útil de 14 casas unifamiliares en \emph{University City}. San
Diego, California. Año 1990. (Ramu Ramanathan, 2002)

Veremos como mostrar los datos, generar diagramas de dispersión,
realizar un ajuste por MCO, y operar con series de datos y con
parámetros estimados. Al final de la práctica aparece el guión
completo con el código que evita trabajar con los menús en modo
gráfico (que es el peor modo de trabajar con Gretl).


\begin{enumerate}
\item Objetivo
\label{sec:org7a81a57}
\begin{enumerate}
\item Reproducir el primer ejemplo de regresión visto en clase.

\item Mostrar datos.
\item Generar gráficos.
\item Guardar un modelo para poder consultarlo más tarde.
\item Recuperar valores ajustados, errores estimados, etc.
\end{enumerate}

\item Carga de datos
\label{sec:orgc67a109}
\textbf{\emph{\texttt{Archivo -{}-> Abrir datos -{}-> Archivo de muestra}}} y en la pestaña
\textbf{\emph{\texttt{Ramanathan}}} seleccione \texttt{data3-1}.

{\vspace{0pt} \footnotesize \color{gray!70!black}
\emph{o bien teclee en linea de comandos}:
\begin{verbatim}
open data3-1
\end{verbatim}
}
\end{enumerate}

\subsection{Actividad 1 - mostrar datos}
\label{sec:org4003853}
\begin{enumerate}
\item Visualice los datos de precios y tamaños de las casas
\label{sec:org430cd84}
\begin{itemize}
\item En la ventana principal de \href{https://gretl.sourceforge.net/es.html}{Gretl}, marque con el ratón ambas
variables: \texttt{price}, \texttt{sqrt}.
\item ``Pinche'' sobre ellas con el botón derecho del ratón.
\item Seleccione \textbf{\emph{\texttt{mostrar valores}}} del menú desplegable que se ha
abierto al pinchar.

{\vspace{1pt} \footnotesize \color{gray!70!black}
\emph{o bien teclee en linea de comandos}:
\begin{verbatim}
print -o price sqft
\end{verbatim}
}
\end{itemize}

\vspace{-3pt}

\item Ayuda
\label{sec:org1d38316}
Para consultar la documentación sobre cualquier comando, puede emplear
el menú desplegable \textbf{\emph{\texttt{Ayuda}}} que aparece arriba, a la derecha de la
ventana principal de \href{https://gretl.sourceforge.net/es.html}{Gretl}.
\begin{itemize}
\item \textbf{\emph{\texttt{Ayuda -> Guía de Instrucciones}}} y ``pinche'' sobre \textbf{\emph{\texttt{print}}}
\end{itemize}

{\vspace{0pt} \footnotesize \color{gray!70!black}
\emph{o bien teclee en linea de comandos}: \texttt{help print}
}
\end{enumerate}

\subsection{Actividad 2 - diagrama de dispersión}
\label{sec:org805554b}
\begin{enumerate}
\item Scatter plot
\label{sec:org7382860}
\begin{itemize}
\item Marque \texttt{price} y \texttt{sqrt} (pulsando \texttt{ctrl} y pinchando con el botón
derecho del ratón sobre ellas). Elija \textbf{\emph{\texttt{Gráfico de dos variables
  XY}}}
\item Seleccione \texttt{sqft} como variable del eje X

{\vspace{1pt} \footnotesize \color{gray!70!black}
\emph{o bien teclee en linea de comandos}: \texttt{gnuplot price sqft}
}
\end{itemize}

\item Guardar gráfico como \emph{icono} para editarlo más tarde
\label{sec:org4a6153b}
\begin{itemize}
\item ``Pinche'' con el botón derecho sobre la ventana del gráfico.
\item Seleccione \textbf{\emph{\texttt{Guardar a sesión como icono}}}

{\vspace{1pt} \footnotesize \color{gray!70!black} \color{gray!70!black}
\emph{o bien teclee en linea de comandos}:
\begin{verbatim}
RectaDeRegresion <- gnuplot price sqft
\end{verbatim}
(\texttt{RectaDeRegresion} \emph{es el nombre con el que se guardará el icono})
}
\end{itemize}

En la zona inferior izquierda de la ventana principal puede ver una
serie de iconos. Uno de ellos es la \textbf{\emph{\texttt{vista de iconos de sesión}}}.
\end{enumerate}

\subsection{Actividad 3 - Ajuste por MCO}
\label{sec:org9bebcb7}

\begin{enumerate}
\item Ajuste por MCO el modelo de regresión visto en clase
\label{sec:org2b28d1b}
\begin{itemize}
\item Estime el modelo mediante los menús desplegables: \textbf{\emph{\texttt{Modelo ->
  Mínimos Cuadrados Ordinarios}}}; indique a \href{https://gretl.sourceforge.net/es.html}{Gretl} el regresando y los
regresores y pulse \textbf{\emph{\texttt{Aceptar}}}.

{\vspace{1pt} \footnotesize \color{gray!70!black} \color{gray!70!black}
\emph{o bien teclee en linea de comandos}:
\begin{verbatim}
ols price 0 sqft
\end{verbatim}
(\emph{el cero} \texttt{0} \emph{indica el término constante}: \texttt{const})
}

\item ``Pinche'' \textbf{\emph{\texttt{Archivo}}} en la ventana del modelo estimado y
seleccione \textbf{\emph{\texttt{guardar como un icono y cerrar}}}

{\vspace{1pt} \footnotesize \color{gray!70!black} \color{gray!70!black}
\emph{o bien teclee en linea de comandos}:
\begin{verbatim}
Regresion  <- ols price 0 sqft
\end{verbatim}
}

\item Recupere el modelo ``pinchando'' sobre su icono

{\vspace{1pt} \footnotesize \color{gray!70!black} \color{gray!70!black}
\emph{o teclee en linea de comandos el nombre que ha dado al icono
  seguido de} \texttt{.show}, \emph{es decir}:
\begin{verbatim}
Regresion.show
\end{verbatim}
}
\end{itemize}
\end{enumerate}

\subsection{Actividad 4 - Recuperar los valores ajustados}
\label{sec:orgfd6352e}
\begin{enumerate}
\item Recuperemos los valores ajustados
\label{sec:org7f2a2a4}
\begin{itemize}
\item Desde la ventana del modelo ajustado (recupérese con su icono),
``pinche'' en \textbf{\emph{\texttt{guardar -> valores estimados}}}. Elija como nombre
\texttt{phat} (puede añadir una descripción de la variable). Pulse en
\textbf{\emph{\texttt{Aceptar}}}
\item Repita para guardar los \texttt{residuos} con el nombre \texttt{ehat}

{\vspace{1pt} \footnotesize \color{gray!70!black}
\emph{o bien teclee en linea de comandos}:
\begin{verbatim}
series phat =  $yhat
series ehat =  $uhat
\end{verbatim}
}
\end{itemize}

\item Mostremos las variables \texttt{price}, \texttt{sqft}, \texttt{phat} y \texttt{ehat}
\label{sec:orgf60eddc}
\begin{itemize}
\item Marque las 4 variables (\texttt{ctrl} y ``pinchar'' con el botón derecho) y elija
\textbf{\emph{\texttt{mostrar valores}}}

{\vspace{1pt} \footnotesize \color{gray!70!black}
\emph{o bien teclee en linea de comandos}: 
\begin{verbatim}
print -o price sqft phat ehat
\end{verbatim}
}
\end{itemize}
\end{enumerate}

\subsection{Actividad 5 - Otras formas de recuperar el ajuste}
\label{sec:org3128734}
\begin{itemize}
\item \texttt{phat2}: restar a los precios los errores

Desde la ventana del modelo: \textbf{\emph{\texttt{Guardar -> Definir una nueva
  variable}}} y teclee:
\texttt{phat2 = price - ehat}

{\vspace{1pt} \footnotesize \color{gray!70!black} \color{gray!70!black}
\emph{o bien teclee en linea de comandos}:
\begin{verbatim}
series phat2 = price - ehat
\end{verbatim}
}

\item \texttt{phat2}: Cálculo \emph{``chapucero''}: 52.351 + 0.139 \emph{sqft}

\textbf{\emph{\texttt{Guardar -> Definir una nueva variable}}} y teclee:
{\vspace{1pt} \footnotesize \color{gray!70!black} \color{gray!70!black}
\begin{verbatim}
series phat3 = 52.351 + 0.139*sqft
\end{verbatim}
}

\item \texttt{phat2}: Cálculo correcto: \Estmc{\beta_1} + \Estmc{\beta_2} \emph{sqft}

\textbf{\emph{\texttt{Guardar -> Definir una nueva variable}}} y teclee:
{\vspace{1pt} \footnotesize \color{gray!70!black} \color{gray!70!black}
\begin{verbatim}
series phat4 = $coeff[1] + $coeff[2]*sqft
\end{verbatim}
}
o bien:
{\vspace{1pt} \footnotesize \color{gray!70!black} \color{gray!70!black}
\begin{verbatim}
series phat5 = $coeff(const) + $coeff(sqft)*sqft
\end{verbatim}
}

\item Visualice los valores como ya hizo en la Actividad 1 de más arriba. ¿Hay diferencias?

{\vspace{1pt} \footnotesize \color{gray!70!black} \color{gray!70!black}
\emph{o bien teclee en linea de comandos}:
\begin{verbatim}
print -o price phat phat2 phat3 phat4 
\end{verbatim}
}
\end{itemize}

\vspace{10pt}
\noindent
\textbf{Código completo de la práctica} \texttt{EjPvivienda.inp}
\vspace{10pt}
\lstinputlisting{scripts/EjPvivienda.inp}
\clearpage

\section{Plano de regresión}
\label{sec:orgbccd1a1}
\begin{center}
\begin{tabular}{ll}
Guión: & \href{https://github.com/mbujosab/Ectr/tree/master/Practicas/Gretl/scripts/PlanoRegresion.inp}{PlanoRegresion.inp}\\[0pt]
\end{tabular}
\end{center}


\begin{itemize}
\item Carga de datos. Abra los menús desplegables: \textbf{\emph{\texttt{Archivo -{}-> Abrir
  datos -{}-> Archivo de muestra}}} y en la pestaña \textbf{\emph{\texttt{POE 4th ed.}}}
seleccione \texttt{andy}.

{\vspace{1pt} \footnotesize \color{gray!70!black}
\emph{o bien teclee en linea de comandos}:
\begin{verbatim}
open andy.gdt
\end{verbatim}
}

\item Ajuste por MCO las ventas a los precios y gastos en publicidad:
\textbf{\emph{\texttt{Modelo -> Mínimos Cuadrados Ordinarios}}}; indique a \href{https://gretl.sourceforge.net/es.html}{Gretl} el
regresando y los regresores y pulse \textbf{\emph{\texttt{Aceptar}}}.

{\vspace{1pt} \footnotesize \color{gray!70!black} \color{gray!70!black}
\emph{o bien teclee en linea de comandos}:
\begin{verbatim}
ols sales 0 price advert
\end{verbatim}
(\emph{el cero} \texttt{0} \emph{indica el término constante}: \texttt{const})
}

\item ``Pinche'' \textbf{\emph{\texttt{Archivo}}} en la ventana del modelo estimado y
seleccione \textbf{\emph{\texttt{guardar como un icono y cerrar}}}

{\vspace{1pt} \footnotesize \color{gray!70!black} \color{gray!70!black}
\emph{o bien teclee en linea de comandos}:
\begin{verbatim}
Regresion  <- ols sales 0 price advert
\end{verbatim}
}

\item Observe el \emph{plano de regresión}: abra la ventana del modelo ajustado
y pinche en \textbf{\emph{\texttt{Gráficos -> Gráfico de variable estimada y observada
  -> Contra price y advert}}}

{\vspace{1pt} \footnotesize \color{gray!70!black} \color{gray!70!black}
}
\end{itemize}

\vspace{10pt}
\noindent
\textbf{Código completo de la práctica} \texttt{PlanoRegresion.inp}
\vspace{10pt}
\lstinputlisting{scripts/PlanoRegresion.inp}
\clearpage


\section{Simulación del ejemplo del precio de las viviendas con tres regresores}
\label{sec:org431fd69}
\begin{center}
\begin{tabular}{ll}
Guión: & \href{https://github.com/mbujosab/Ectr/tree/master/Practicas/Gretl/scripts/SimuladorEjPvivienda.inp}{SimuladorEjPvivienda.inp}\\[0pt]
\end{tabular}
\end{center}

En este ejercicio con \href{https://gretl.sourceforge.net/es.html}{Gretl} generaremos unos datos simulados para realizar regresiones MCO.

\begin{enumerate}
\item Simulamos series de datos de 500 observaciones

\begin{itemize}
\item \textbf{\emph{\texttt{Archivo -> Nuevo conjunto de datos}}}, e indicamos el número
de observaciones: 1500. Marcamos \emph{de sección cruzada} y
continuamos adelante. Dejamos sin marcar \texttt{empezar a introducir
     los valores de los datos} y pulsamos \texttt{Aceptar}.

\item o bien:
\begin{verbatim}
nulldata 1500
\end{verbatim}
\end{itemize}

\item Generamos tres variables: \texttt{S} con distribución uniforme (35, 120);
\texttt{D} con distribución Chi cuadrado con 5 grados de libertad que
vamos a multiplicar por 3 y \texttt{U} con distribución Normal de media 0
y desviación típica 40

\begin{itemize}
\item \textbf{\emph{\texttt{Añadir -> Variable aleatoria}}} y se elige para cada
variable el tipo de distribución, los valores de los parámetros y
el nombre de la variable

\item o bien
\begin{verbatim}
series S = randgen(U, 35, 120)
series D = randgen(X, 5) * 3
series U = randgen(N, 0, 40)
\end{verbatim}
\end{itemize}

\item Simulamos los precios según el modelo \(\Vect{p} = 300\Vect{1} +
   5\Vect{s} - 2\Vect{d} + \Vect{u}\)

\begin{itemize}
\item \textbf{\emph{\texttt{Añadir -> Definir nueva variable}}} y tecleamos: \texttt{P = 300 + 5*S - 2*D + U}

\item O bien 
\begin{verbatim}
series P = 300 + 5*S - 2*D + U
\end{verbatim}
\end{itemize}

\item Observe los estadísticos de las variables de modelo. En particular,
¿son ortogonales \texttt{S} y \texttt{D} respecto al regresor constante \Vect{1}?

\begin{itemize}
\item \textbf{\emph{\texttt{Ver -> Estadísticos principales}}} con el ratón marcamos \texttt{D},
\texttt{S} y \texttt{P}.

\item O bien 
\begin{verbatim}
summary P S D
\end{verbatim}
\end{itemize}

\item El valor absoluto de la correlación entre \texttt{D} y \texttt{S} ¿es grande o
pequeño?

\begin{itemize}
\item \textbf{\emph{\texttt{Ver -> Matriz de correlación}}} y selecciones \texttt{D} y \texttt{S}.

\item O bien 
\begin{verbatim}
corr S D
\end{verbatim}
\end{itemize}

\item Observe los diagramas de dispersión de \texttt{P} con \texttt{S}, el de \texttt{P} con
\texttt{D}, y el de los regresores \texttt{S} y \texttt{D}:

\begin{itemize}
\item \textbf{\emph{\texttt{Ver -> Gráficos -> Gráfico X-Y (Scatter)}}}. Indicamos \texttt{S} como \texttt{variable del eje x}. Indicamos \texttt{P} como \texttt{variable del eje y}.

\item \textbf{\emph{\texttt{Ver -> Gráficos -> Gráfico X-Y (Scatter)}}}. Indicamos \texttt{D} como \texttt{variable del eje x}. Indicamos \texttt{P} como \texttt{variable del eje y}.

\item \textbf{\emph{\texttt{Ver -> Gráficos -> Gráfico X-Y (Scatter)}}}. Indicamos \texttt{S} como \texttt{variable del eje x}. Indicamos \texttt{D} como \texttt{variable del eje y}.

\item O bien 
\begin{verbatim}
scaterPS <- gnuplot P S
scaterPD <- gnuplot P D
scaterDS <- gnuplot D S
\end{verbatim}

Nótese como dichos diagramas reflejan las correlaciones entre las distintas variables del modelo.
\end{itemize}

\item Ajuste por MCO \texttt{P} empleando \texttt{S} y \texttt{D}:

\begin{itemize}
\item pinche en \textbf{\emph{\texttt{Modelo -> Mínimos Cuadrados Ordinarios}}} y selecciones \texttt{D} y \texttt{S}.

\begin{itemize}
\item Elija \texttt{P} como regresando o "variable dependiente" (marque la opción \texttt{Selección por defecto}).

\item Elija \texttt{S} y \texttt{D} como regresores o "Variables independientes". Pinche en \texttt{Aceptar}.

\item En la ventana del modelo pinche en \textbf{\emph{\texttt{Archivo -> Guardar como icono y cerrar}}}.

\item Con el botón derecho del ratón, pinche sobre el icono del
modelo estimado y seleccione \texttt{Añadir a la tabla de modelos}.
\end{itemize}

\item O bien 
\begin{verbatim}
ModeloCompleto <- ols P 0 S D
modeltab add
\end{verbatim}
\end{itemize}

Observe que el ajuste recupera aproximadamente los valores de los
parámetros del modelo simulado.

Observe el \emph{plano de regresión}: abra la ventana del modelo
ajustado y pinche en \textbf{\emph{\texttt{Gráficos -> Gráfico de variable estimada y
   observada -> Contra S y D}}}
\end{enumerate}



\begin{itemize}
\item \textbf{Tarea adicional} Observe que el papel del parámetro \(\beta_1\)
que acompaña al término constante es equilibrar los valores
medios a ambos lados de la ecuación; es decir, asegurar que
\[\media{\Vect{p}}=\Estmc{\beta_1}+\Estmc{\beta_2}\media{\Vect{s}}+\Estmc{\beta_3}\media{\Vect{d}}\]
Verifique que efectivamente
\[\Estmc{\beta_1}=\media{\Vect{p}}-\Estmc{\beta_2}\media{\Vect{s}}-\Estmc{\beta_3}\media{\Vect{d}}\]

\begin{itemize}
\item Hágalo empleando una calculadora o indique el cálculo en el guión de Gretl con
\begin{verbatim}
## TAREA ADICIONAL ####
scalar AjusteMediasC = mean(P) - $coeff(S)*mean(S) - $coeff(D)*mean(D)
scalar beta1HatC     = $coeff(const)
######################	 
\end{verbatim}

\emph{Otra manera de aludir a los betas estimados es generar una
matriz columna con los betas: \texttt{beta = \$coeff}, de manera que
\texttt{beta[1,1]}, \texttt{beta[2,1]} y \texttt{beta[3,1]} son \(\Estmc{\beta_1}\),
\(\Estmc{\beta_2}\) y \(\Estmc{\beta_3}\) respectivamente.}
\end{itemize}
\end{itemize}

\subsection{Omisión de regresores}
\label{sec:orgb0c6c8b}
\begin{enumerate}
\item Si se omite el regresor \texttt{D} del ajuste ¿qué esperaría que ocurra
con el parámetro asociado a la cte?

\begin{itemize}
\item pinche en \textbf{\emph{\texttt{Modelo -> Mínimos Cuadrados Ordinarios}}}

\begin{itemize}
\item En la lista de variables dependientes, pinche sobre \texttt{D} y pulse la flecha roja para eliminar dicha variable del
modelo. Pinche en \texttt{Aceptar}.

\item la ventana del modelo pinche en \textbf{\emph{\texttt{Archivo -> Guardar como icono y cerrar}}}.

\item añada este modelo a la \texttt{Tabla de modelos}.
\end{itemize}

\item O bien 
\begin{verbatim}
ModeloSinD <- ols P 0 S
modeltab add
\end{verbatim}
\end{itemize}

¿Confirman los resultados lo esperado respecto a los parámetros estimados?

\begin{itemize}
\item \textbf{Tarea adicional} Recuerde que el papel del parámetro \(\beta_1\)
que acompaña al término constante es equilibrar los valores
medios a ambos lados de la ecuación; en este caso,
\[\media{\Vect{p}}=\Estmc{\beta_1}+\Estmc{\beta_2}\media{\Vect{s}}\]
Verifique que efectivamente
\[\Estmc{\beta_1}=\media{\Vect{p}}-\Estmc{\beta_2}\media{\Vect{s}}\]

\begin{itemize}
\item Hágalo empleando una calculadora o indique el cálculo en el guión de Gretl con
\begin{verbatim}
## TAREA ADICIONAL ####
scalar AjusteMediasNoD = mean(P) - $coeff(S)*mean(S)
scalar beta1HatNoD = $coeff(const)
######################	 
\end{verbatim}
\end{itemize}
\end{itemize}

\item Repita el experimento, pero esta vez incluyendo \texttt{D} y excluyendo \texttt{S}

\begin{itemize}
\item pinche en \textbf{\emph{\texttt{Modelo -> Mínimos Cuadrados Ordinarios}}}

\begin{itemize}
\item Elimine de la lista de regresores la variable \texttt{S}, e incluya \texttt{D} como regresor. Pinche en \texttt{Aceptar}.

\item la ventana del modelo pinche en \textbf{\emph{\texttt{Archivo -> Guardar como icono y cerrar}}}.

\item añada este modelo a la \texttt{Tabla de modelos}.
\end{itemize}

\item O bien 
\begin{verbatim}
ModeloSinS <- ols P 0 D
modeltab add
\end{verbatim}
donde \texttt{modeltab show} muestra la tabla de modelos ajustados.
\end{itemize}

¿Confirman los resultados lo esperado respecto a los parámetros estimados?

\begin{itemize}
\item \textbf{Tarea adicional} Aquí \(\beta_1\) asegura que
\[\media{\Vect{p}}=\Estmc{\beta_1}+\Estmc{\beta_2}\media{\Vect{d}}\]
Verifique que efectivamente
\[\Estmc{\beta_1}=\media{\Vect{p}}-\Estmc{\beta_2}\media{\Vect{d}}\]

\begin{itemize}
\item Hágalo empleando una calculadora o indique el cálculo en el guión de Gretl con
\begin{verbatim}
## TAREA ADICIONAL ####
scalar AjusteMediasNoS = mean(P) - $coeff(D)*mean(D)
scalar beta1HatNoD = $coeff(const)
######################	 
\end{verbatim}
\end{itemize}
\end{itemize}
\end{enumerate}

\subsection{Proyección omitiendo las constantes}
\label{sec:orge76fe4b}

\begin{itemize}
\item ¿Y si calculamos la proyección ortogonal de los precios sobre \texttt{S}
y \texttt{D} omitiendo la constante? ¿Qué espera que ocurra con los valores medios de \texttt{P}
y del ajuste? Verifique su respuesta con el ordenador.

\begin{itemize}
\item pinche en \textbf{\emph{\texttt{Modelo -> Mínimos Cuadrados Ordinarios}}}

\begin{itemize}
\item Elija \texttt{S} y \texttt{D} como regresores, pero elimine la constante de la
lista (\emph{¡algo que jamás debe hacer en sus
estimaciones!}). Pinche en \texttt{Aceptar}.

\item la ventana del modelo pinche en \textbf{\emph{\texttt{Archivo -> Guardar como icono y cerrar}}}.

\item añada este modelo a la \texttt{Tabla de modelos}.
\end{itemize}

\item O bien 
\begin{verbatim}
ModeloSin1 <- ols P S D
modeltab add
\end{verbatim}
\end{itemize}

¿Confirman los resultados lo esperado respecto a los parámetros estimados?

\begin{itemize}
\item \textbf{Tarea adicional} Aquí no hay un término constante, así que
\[\media{\Vect{p}} \ne \Estmc{\beta_1}\media{\Vect{s}} + \Estmc{\beta_2}\media{\Vect{d}}\]
El equilibro se da añadiendo la media de los errores (que por no
existir término constante es distinta de cero).  Verifique que
efectivamente
\[\media{\Vect{p}} = \Estmc{\beta_1}\media{\Vect{s}} + \Estmc{\beta_2}\media{\Vect{d}} + \media{\res}\]

\begin{itemize}
\item Hágalo empleando una calculadora o indique el cálculo en el guión de Gretl con
\begin{verbatim}
## TAREA ADICIONAL ####
scalar MediaLadoIzdo = mean(P)
scalar MediaLadoDcho = $coeff(S)*mean(S) + $coeff(D)*mean(D) + mean($uhat)
######################	 
\end{verbatim}
donde \texttt{\$uhat} son los residuos de la  última regresión realizada.
\end{itemize}

Este ajuste sin constante, \textbf{y que por tanto no debemos llamar
\emph{regresión MCO}}, no logra equiparar correctamente los valores
medios de ambos lados de la ecuación. Los únicos elementos
disponibles para intentar dicho ajuste han sido los vectores de
datos \texttt{S} y \texttt{D}, que tienen medias distintas de cero. Pero
simultáneamente dichos vectores son necesarios para lograr el
ajuste de las pendientes. Por ello los resultados al omitir la
constante pueden ser desastrosos.
\end{itemize}
\end{itemize}

\subsection{``Ortogonalizando'' los regresores no constantes respecto de las constantes}
\label{sec:orgf0ecda3}

\begin{enumerate}
\item ¿Hay alguna forma de “ortogonalizar” \texttt{S} y \texttt{D} respecto de la
constante?\ldots{} ¡Si!, basta restar las medias (generar nuevos
regresores en desviaciones respecto a su media):

\begin{itemize}
\item \textbf{\emph{\texttt{Añadir -> Definir nueva variable}}} y escribimos \texttt{series S0 =
    S - mean(S)}
\item \textbf{\emph{\texttt{Añadir -> Definir nueva variable}}} y escribimos \texttt{series D0 =
    D - mean(D)}
\item O bien 
\begin{verbatim}
series S0 = S - mean(S)
series D0 = D - mean(D)
\end{verbatim}
\end{itemize}

\item Verifique que la media de \texttt{S0} y \texttt{D0} es cero. Ajuste por MCO los siguientes modelos:
\[P = \beta_1 + \beta_2 S0 + \beta_3 D0\]
y
\[P = \beta_2 S0 + \beta_3 D0\]

Compare los resultados entre ambos y con el modelo completo original.

\begin{verbatim}
ModeloCompEnDesviaciones <- ols P 0 S0 D0
modeltab add
ModeloEnDesviaciones <- ols P S0 D0
modeltab add
modeltab show
\end{verbatim}

\begin{itemize}
\item Compare la estimación de \(\beta_1\) del \texttt{ModeloCompEnDesviaciones}
con la media de \texttt{P}.

\item ¿Se le ocurre alguna explicación para este resultado?

\item Observe en la tabla de modelos las similitudes del primer modelo
en desviaciones con las obtenidas con el modelo completo
original\ldots{} ¡Todo es idéntico salvo la estimación de \(\beta_1\)!

\item Observe que aunque las pendientes estimadas toman los mismos
valores en ambos modelos en desviaciones, la incertidumbre
asociada en el caso del modelo en desviaciones sin contante es
mucho mayor
\end{itemize}
\end{enumerate}

\vspace{10pt}
\noindent
\textbf{Código completo de la práctica} \texttt{SimuladorEjPvivienda.inp}
\vspace{10pt}
\lstinputlisting{scripts/SimuladorEjPvivienda.inp}
\clearpage


\section{Repetición del ejemplo pero introduciendo correlación entre regresores}
\label{sec:org9331309}
\begin{center}
\begin{tabular}{ll}
Guión: & \href{https://github.com/mbujosab/Ectr/tree/master/Practicas/Gretl/scripts/SimuladorEjPvivienda2.inp}{SimuladorEjPvivienda2.inp}\\[0pt]
\end{tabular}
\end{center}

En la simulación de la práctica anterior no había una correlación muy
fuerte entre \texttt{S} y \texttt{D}. En ejemplos reales podemos encontrar una mayor
riqueza de interrelaciones entre regresores.  Para experimentar las
consecuencias, vamos a repetir todo el ejercicio pero introduciendo un
factor común entre \texttt{S} y \texttt{D} que genere correlación entre ambos
regresores.

\begin{description}
\item[{Los datos}] Vamos a variar el modo de generar los datos respecto a la práctica anterior.

\begin{itemize}
\item Genere un factor común \texttt{C} con distribución normal de esperanza 40
y desviación típica 20: \textbf{\emph{\texttt{Añadir -> Definir nueva variable}}} y
escribimos \texttt{series C = randgen(N, 40, 20)}

\item Genere un vector de datos de superficies con distribución uniforme
entre 20 y 70 y súmele \texttt{C}: \textbf{\emph{\texttt{Añadir
    -> Definir nueva variable}}} y escribimos \texttt{series S = randgen(U, 20, 70) + C}

\item Genere un vector de tiempos de viaje con distribución Chi cuadrado
con 8 grados de libertad, y súmele 100 y réstele C: \textbf{\emph{\texttt{Añadir ->
    Definir nueva variable}}} y escribimos \texttt{series D = randgen(X, 8) +
    100 - C}

\item Genere un vector de perturbaciones con distribución Normal de
media 0 y desviación típica 40. \textbf{\emph{\texttt{Añadir -> Definir nueva
    variable}}} y escribimos \texttt{series U = randgen(N, 0, 40)}

\item O bien
\begin{verbatim}
nulldata 1500
series C = randgen(N, 40, 20)
series S = randgen(U, 20, 70) + C
series D = randgen(X, 8) + 100 - C
series U = randgen(N, 0, 40)
\end{verbatim}
\end{itemize}
\end{description}


\begin{description}
\item[{El modelo}] que de nuevo simulamos es: \(\Vect{p} = 300\Vect{1} +
  5\Vect{s} - 2\Vect{d} + \Vect{u}\)

\begin{itemize}
\item \textbf{\emph{\texttt{Añadir -> Definir nueva variable}}} y tecleamos: \texttt{P = 300 + 5*S - 2*D + U}

\item O bien 
\begin{verbatim}
series P = 300 + 5*S - 2*D + U
\end{verbatim}
\end{itemize}
\end{description}


\begin{description}
\item[{Actividades}] Repita todas las actividades de la práctica anterior pero con estos
nuevos datos simulados.  Preste especial atención a cuáles son las
diferencias entre esta simulación y la anterior sin correlación
entre \texttt{S} y \texttt{D}.

Como ya están descritas en la práctica anterior, a continuación tan solo mostrare el guión modificado.
\end{description}
\clearpage
\noindent
\textbf{Código completo de la práctica} \texttt{SimuladorEjPvivienda2.inp}
\vspace{10pt}
\lstinputlisting{scripts/SimuladorEjPvivienda2.inp}
\clearpage


\section{Más sobre la ortogonalización de regresores}
\label{sec:org35c3827}
\begin{center}
\begin{tabular}{ll}
Guión: & \href{https://github.com/mbujosab/Ectr/tree/master/Practicas/Gretl/scripts/SimuladorEjPvivienda3.inp}{SimuladorEjPvivienda3.inp}\\[0pt]
\end{tabular}
\end{center}

En el ejemplo anterior hemos visto que al generar nuevos regresores
\texttt{S0} y \texttt{P0} que son las componentes de \texttt{S} y \texttt{P} perpendiculares a las
constantes, y usarlos en sustitución de los regresores originales,
hemos logrado nuevos ajustes en los que incluir u omitir el regresor
constante no cambia el valor de las pendientes estimadas.

\subsection{Empleando regresores no constantes que son perpendiculares entre si}
\label{sec:org24a0da6}

Veamos si es posible generalizar este resultado, es decir, si al
emplear como regresor la componente de \texttt{S} perpendicular a \texttt{P} y la
constante, podemos ajustar \texttt{P} con tres regresores perpendiculares
entre sí, de tal manera que omitir cualesquiera de los regresores no
afecta a los parámetros estimados correspondientes a los regresores
que sí se mantienen en el ajuste.

\begin{itemize}
\item Simulamos el modelo de la práctica anterior
\end{itemize}
\begin{verbatim}
nulldata 1500
series C = randgen(N, 40, 20)
series S = randgen(U, 20, 70) + C
series D = randgen(X, 8) + 100 - C
series U = randgen(N, 0, 40)
series P = 300 + 5*S - 2*D + U
\end{verbatim}

\begin{itemize}
\item Ajustamos los primeros tres modelos de la práctica anterior y los
guardamos como iconos.
\end{itemize}

\begin{verbatim}
ModeloCompleto <- ols P 0 S D
modeltab add    
ModeloSinD <- ols P 0 S
modeltab add
ModeloSinS <- ols P 0 D
modeltab add
ModeloSin1 <- ols P S D
modeltab add
\end{verbatim}

\begin{itemize}
\item El último ajuste es una proyección ortogonal sin término constante
(no una regresión). Calculemos el coeficiente de determinacion y el
coeficiente de determinacion ajustado calculando explícitamente su
fórmula.
\end{itemize}

\begin{verbatim}
# calculo del coeficiente de determinacion
scalar R2Sin1 = 1 - $ess/(($T-1) * var(P))
scalar R2AjustadoSin1 = 1 - ($T-1)/($T-2)*(1 - R2Sin1)
\end{verbatim}

\subsubsection{Regresiones auxiliares}
\label{sec:org03f4d08}

\begin{itemize}
\item Calculamos la componente de \texttt{D} que es ortogonal a los vectores
constantes y la denominamos \texttt{D0}
\end{itemize}

\begin{verbatim}
# ortogonalizamos D respecto de las constantes
series D0 = D - mean(D)
# o de manera equivalente
ols D 0 
series D0 = $uhat

\end{verbatim}

\begin{itemize}
\item Calculamos la componente de \texttt{S} que es ortogonal a los vectores
constantes y a \texttt{D} y la denominamos \texttt{SpD}
\end{itemize}

\begin{verbatim}
# ortogonalizamos S respecto de las constantes y D
ols S 0 D
series SpD = $uhat
\end{verbatim}

\subsubsection{Ajuste de \texttt{P} con los nuevos regresores}
\label{sec:org6cdea41}

\begin{itemize}
\item Mediante una regresión ajustamos por MCO \texttt{P} a los nuevos regresores
\end{itemize}

\begin{verbatim}
ModeloCompleto2   <- ols P 0 SpD D0
modeltab add
\end{verbatim}

\begin{itemize}
\item Mediante una regresión ajustamos por MCO \texttt{P} únicamente al regresor constante
\end{itemize}

\begin{verbatim}
ModeloSolo1       <- ols P 0
modeltab add
\end{verbatim}

\begin{itemize}
\item Mediante una proyección ajustamos \texttt{P} a \texttt{SpD}
\end{itemize}

\begin{verbatim}
ModeloSoloS       <- ols P SpD
modeltab add
\end{verbatim}

\begin{itemize}
\item Mediante una proyección ajustamos \texttt{P} a \texttt{D0}
\end{itemize}

\begin{verbatim}
ModeloSoloD       <- ols P D0
modeltab add
\end{verbatim}

\begin{itemize}
\item Compare los resultados de los distintos ajustes: los valores de los
parámetros y el ajuste medido con el \(R^2\) corregido
\end{itemize}

\begin{verbatim}
modeltab show
\end{verbatim}

\subsection{Ajustes aún más curiosos}
\label{sec:org65729cf}
\begin{verbatim}
## Dos ajustes curiosos 
\end{verbatim}
En los siguientes ajustes lo que se mantiene es el valor del parámetro
correspondiente la constante del modelo original (aunque no aparezca
explícitamente ningún regresor constante en el ajuste\ldots{} ahí radica la
curiosidad).

\subsubsection{Una primera proyección auxiliar}
\label{sec:orgd8782c3}
Lo primero es obtener la componente del vector \Vect{1} que es
perpendicular a \texttt{S} y a \texttt{D} mediante una primera proyección auxiliar
(no la llamaré regresión, pues el espacio sobre el que proyectamos no
contiene los vectores constantes). Llamaré \texttt{cte} a dicha componente
(pero obviamente no es un vector constante):
\begin{verbatim}
ols 0 S D                # proyeccion del vector 1 sobre S y D
series cte = $uhat       # componente del vector 1 que es perpendicular a S y a D
\end{verbatim}


\subsubsection{Primer ajuste curioso}
\label{sec:orgdb80fae}
En una nueva \emph{proyección} auxiliar obtenemos la componente de \texttt{S} que
es perpendicular a \texttt{cte} y a \texttt{D}, que llamaré \texttt{SpD} (componente de \texttt{S}
perpendicular a \texttt{D} y a \texttt{cte}).
\begin{verbatim}
##################### Primer ajuste curioso ##################
ols S cte D
series SpD = $uhat       # componente de S que es perpendicular a "cte" y a D
\end{verbatim}
Y ahora realizamos la \emph{regresión} de \texttt{P} sobre el mismo subespacio que
en modelo original, pero empleando tres regresores que son ortogonales
entre si, y tales que el parámetro de \texttt{cte} es igual al
correspondiente a la constante del modelo original y el de \texttt{SpD} al
del parámetro de \texttt{S} en la \emph{proyección} de \texttt{P} sobre \texttt{S} y \texttt{D}
(omitiendo la constante).
\begin{verbatim}
# regresion MCO que mantiene la pendiente de S del ajuste sin cte
ModeloCompletoAlternativo1 <- ols P cte SpD D 
modeltab add 
\end{verbatim}

\subsubsection{Segundo ajuste curioso}
\label{sec:orgb5f934e}
Alternativamente podemos dar los pasos análogos, pero usando la
\emph{proyección} auxiliar de \texttt{D} sobre \texttt{cte} y \texttt{S}.
\begin{verbatim}
##################### Segundo ajuste curioso ##################
ols D cte S
series DpS = $uhat       # componente de D que es perpendicular a cte y a S
## regresion MCO que mantiene la pendiente de D del ajuste sin cte
ModeloCompletoAlternativo2 <- ols P cte DpS S 
modeltab add  
modeltab show
\end{verbatim}

Compare los resultados en la tabla de modelos.

\clearpage
\noindent
\textbf{Código completo de la práctica} \texttt{SimuladorEjPvivienda3.inp}

\lstinputlisting{scripts/SimuladorEjPvivienda3.inp}
\clearpage
\end{document}
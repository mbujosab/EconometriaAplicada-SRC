% Created 2023-11-10 vie 13:30
% Intended LaTeX compiler: pdflatex
\documentclass[11pt]{article}
\usepackage[utf8]{inputenc}
\usepackage[T1]{fontenc}
\usepackage{graphicx}
\usepackage{longtable}
\usepackage{wrapfig}
\usepackage{rotating}
\usepackage[normalem]{ulem}
\usepackage{amsmath}
\usepackage{amssymb}
\usepackage{capt-of}
\usepackage{hyperref}
\usepackage[spanish, ]{babel}
\usepackage[margin=0.5in]{geometry}
\usepackage[svgnames,x11names]{xcolor}
\hypersetup{linktoc = all, colorlinks = true, urlcolor = DodgerBlue4, citecolor = PaleGreen1, linkcolor = SpringGreen4}
\PassOptionsToPackage{hyphens}{url}
\usepackage{nacal}
\usepackage{framed}
\usepackage{listings}
\lstset{language = hansl, %
        showstringspaces = false, % 
        basicstyle = \footnotesize\ttfamily,
        keywordstyle = \color{red!50!black}\ttfamily,
        keywordstyle = [2]\color{green!40!black}\bfseries,
        keywordstyle = [3]\color{cyan!65!black}\ttfamily,
        keywordstyle = [4]\color{green!55!black}\ttfamily,
        commentstyle = \color{blue!70!black}\sffamily,       % 
        stringstyle = \color{magenta!75!black}\ttfamily,
        % backgroundcolor=\color{lightgray},
        % These commands partly sorted by meaning:
        morekeywords = {list, bundle, %
                matrices, strings, bundles, lists, %
                funcerr, hfplot, launch, midasreg, %
                pkg, plot, setopt, % 
                deriv, params, %
                for, foreach, while, 
                },
        morekeywords = [2]{array, defarray, null, %
                BFGSmin, BFGScmax, BFGScmin, %
                bread, bwrite, cdummify, cnumber, %
                cnameget, cnameset, dropcoll, %
                easterday, ecdf, exists, fevd, genseries, %
                getinfo, getkeys, GSSmax, GSSmin, %
                hfdiff, hfldiff, hflags, hflist, instring, %
                isdiscrete, isdummy, isoconv, isodate, %
                jsonget, jsongetb, juldate, kdsmooth, %
                kmeier, kpsscrit, ksetup, linearize, %
                lrcovar, mgradient, mlincomb, mweights, %
                naalen, nlines, NMmax, NMmin, %
                normtest, npcorr, NRmin, numhess, %
                pexpand, pxnobs, qlrpval, %             
                rnameget, rnameset, %
                seasonals, sleep, smplspan, sprintf, %
                square, stringify, strvals, substr, %
                typeof, varnames, xmlget, series},
        % the following for the double-dash options:
        %% moredelim       = [l][\color{green!50!gray}\ttfamily]{--},
        % obsolete former functions:
        deletekeywords = [2]{isnull, rowname, rownames, %
                colname, colnames, xpx },
        % and obsolete former commands:
        deletekeywords = {kalman, shell, sscanf},
        % try to redefine / override the $-accessors (not as group-1-keywords)
        keywordsprefix = {\$\$}, % anything which isn't a single $
        morekeywords = [3]{\$coeff,\$uhat, % to be completed
                \$ahat, \$aic, \$bic, \$chisq,
                \$command, \$compan, \$datatype, \$depvar, \$df,
                \$diagpval, \$diagtest, \$dw, \$dwpval, \$ec,
                \$error, \$ess , \$evals, \$fcast, \$fcse,
                \$fevd, \$Fstat, \$gmmcrit, \$h, \$hausman,
                \$hqc, \$huge, \$jalpha, \$jbeta, \$jvbeta,
                \$lang, \$llt, \$lnl, \$macheps, \$mnlprobs,
                \$model, \$ncoeff, \$nobs, \$nvars, \$obsdate,
                \$obsmajor, \$obsmicro, \$obsminor, \$pd , \$pi,
                \$pvalue, \$qlrbreak, \$rho, \$rsq, \$sample,
                \$sargan, \$sigma, \$stderr, \$stopwatch, \$sysA,
                \$sysB, \$sysGamma, \$sysinfo, \$system, \$T,
                \$t1, \$t2, \$test, \$tmax, \$trsq,
                \$uhat, \$unit, \$vcv, \$vecGamma, \$version,
                \$vma, \$windows, \$xlist, \$xtxinv, \$yhat,
                \$ylist},
        morekeywords = [4]{scalar}, %
        %morekeywords = [5]{randgen,mean,std}, %
    inputencoding = utf8,  % Input encoding
    extendedchars = true,  % Extended ASCII
    texcl         = true,  % Activate LaTeX commands in comments
    literate      =        % Support additional characters
      {á}{{\'a}}1  {é}{{\'e}}1  {í}{{\'i}}1 {ó}{{\'o}}1  {ú}{{\'u}}1
      {Á}{{\'A}}1  {É}{{\'E}}1  {Í}{{\'I}}1 {Ó}{{\'O}}1  {Ú}{{\'U}}1
      {ü}{{\"u}}1  {Ü}{{\"U}}1  {ñ}{{\~n}}1 {Ñ}{{\~N}}1  
}

\lstnewenvironment{hansl-gretl}
{\lstset{language={hansl},basicstyle={\ttfamily\footnotesize},numbers,rame=single,breaklines=true}}
{}
\newcommand{\hansl}[1]{\lstset{language={hansl},basicstyle={\ttfamily\small}}\lstinline{#1}}
\lstset{backgroundcolor=\color{lightgray!20}, }
\author{Marcos Bujosa}
\date{\today}
\title{Lección 9}
\begin{document}

\maketitle
\tableofcontents

\clearpage

\section{Estimación restringida vía mínimos cuadrados restringidos y vía sustitución}
\label{sec:orgde5c535}
\begin{center}
\begin{tabular}{ll}
Guión: & \href{https://github.com/mbujosab/Ectr/tree/master/Practicas/Gretl/scripts/GujaratiEx8-3.inp}{GujaratiEx8-3.inp}\\[0pt]
\end{tabular}
\end{center}

Cargue los datos \texttt{Table\_8.8.gdt} de la pestaña del libro de Gujarati.
\begin{verbatim}
open Table_8.8.gdt
\end{verbatim}

\begin{description}
\item[{Estimación restringida}] Suponga que deseamos estimar el siguiente
modelo en logaritmos proveniente de una función de Cobb-Douglas:
\begin{displaymath}
   \ln \VA{Y} = 
   \beta_1\VAindUno + \beta_2\ln\VA{k} + \beta_3\ln\VA{L} + \per;
\end{displaymath}
pero que deseamos imponer la restricción de rendimientos constantes a
escala, es decir, \(\beta_2+\beta_3=1\). Veamos dos maneras equivalentes
de proceder:

\begin{enumerate}
\item Transforme las variables en logaritmos
\begin{verbatim}
logs GDP Employ Capital
\end{verbatim}
\begin{itemize}
\item Estime por MCO el modelo sin restringir (guarde el el modelo como
icono con el nombre \texttt{U} (unrestricted).
\begin{verbatim}
# modelo 1
U <- ols l_GDP const l_Employ l_Capital
\end{verbatim}
\item Imponga la restricción \(\beta_2+\beta_3=1\). Desde la ventana del
modelo estimado sin restricciones siga los pasos \textbf{\emph{\texttt{Contrastes ->
      Restricciones lineales}}} y teclee \texttt{b[2]+b[3]=1} o bien, en un
guión o la consola teclee
\begin{verbatim}
restrict
 b[2]+b[3]=1
end restrict
\end{verbatim}
Observe los coeficientes estimados resultantes tras imponer la
restricción.
\end{itemize}

\item Defina las variables \textsf{Capital/Employ} y \textsf{GDP/Employ}
y transforme las nuevas variables mediante logaritmos.
\begin{verbatim}
series CE   = Capital/Employ
series GDPE = GDP/Employ
logs CE GDPE
\end{verbatim}
\begin{itemize}
\item Estime por MCO el modelo
\begin{displaymath}
   \ln \VA{Y} = \beta_1\VAindUno + \beta_2 \ln\frac{\VA{k}}{\VA{L}} + \per
\end{displaymath}
y compare los resultados anteriores (los del primer modelo tras
imponer la restricción).
\begin{verbatim}
# modelo 2
R <- ols l_GDPE const l_CE
\end{verbatim}
\end{itemize}
\end{enumerate}

\item[{Contraste de la hipótesis}] Calcule el estadístico \(\Festadistico\)
(en su formulación mediante sumas residuales de los modelos
restringidos y sin restringir) y su \emph{p}-valor para contrastar la
hipótesis de rendimientos constantes a escala. ¿Rechaza la \Hnula al
5\% de significación?
\begin{verbatim}
/* contraste de la F mediante sumas residuales */
scalar r = 1                             # num. restriciones (solo una en este caso)
scalar f = U.$df*(R.$ess-U.$ess)/U.$ess/r
pvalue F r U.$df f
\end{verbatim}
\end{description}

\clearpage
\vspace{10pt}
\noindent
\subsection{Código completo de la práctica \texttt{GujaratiEx8-3.inp}}
\label{sec:org3870d9b}
\vspace{10pt}
\lstinputlisting{scripts/GujaratiEx8-3.inp}
\clearpage


\section{Test de Chow de cambio estructural}
\label{sec:org4b34ea2}
\begin{center}
\begin{tabular}{ll}
Guión: & \href{https://github.com/mbujosab/Ectr/tree/master/Practicas/Gretl/scripts/GujaratiSec8-8.inp}{GujaratiSec8-8.inp}\\[0pt]
\end{tabular}
\end{center}

Cargue los datos \texttt{Table\_8.9.gdt} de la pestaña del libro de Gujarati
con datos para la economía americana del 1970 a 1995.
\begin{verbatim}
open Table_8.9.gdt
\end{verbatim}

Considere el modelo:
\begin{displaymath}
  \VA{Y} = \beta_1\VAindUno + \beta_2\VA{X} + \per,
\end{displaymath}
donde \VA{X} es el ahorro de las familias e \VA{Y} es la renta
disponible.

En el año 1982 se produjo una importante crisis económica. Contraste
si el modelo es idéntico para toda la muestra, o si se produjo un
cambio estructural (use los periodos 1970--1981 y 1982--1995).

\begin{itemize}
\item Estime el modelo restringido (mismos betas para todo el
periodo). Guarde la Suma de los Residuos al Cuadrado (SRC)
\begin{verbatim}
ModeloR       <- ols SAVINGS 0 INCOME --quiet
scalar sse_R  = $ess
\end{verbatim}

\item Estime dos modelos, uno para los 12 primeros datos y otro para los
14 siguientes. Guarde la Suma de los Residuos al Cuadrado (SRC)
conjunta del modelo sin restringir.
\begin{verbatim}
smpl 1 12
ModeloU1      <- ols SAVINGS 0 INCOME --quiet
scalar sse_U1  = $ess

smpl full
smpl 13 26
ModeloU2      <- ols SAVINGS 0 INCOME --quiet
scalar sse_U2  = $ess
\end{verbatim}

\item Calcule el estadístico del contraste de cambio estructural de Chow y
su p-valor.
\begin{verbatim}
scalar sse_U = sse_U1+sse_U2
smpl full
scalar chowtest = ((sse_R-sse_U)/2)/(sse_U/($nobs-4))
pvalue F 2 $nobs-4 chowtest
\end{verbatim}

\item ¿Rechaza que el modelo es el mismo para todo el periodo? ¿o no?
\end{itemize}


\clearpage
\vspace{10pt}
\noindent
\subsection{Código completo de la práctica \texttt{GujaratiSec8-8.inp}}
\label{sec:orgb2b398d}
\vspace{10pt}
\lstinputlisting{scripts/GujaratiSec8-8.inp}
\clearpage
\end{document}
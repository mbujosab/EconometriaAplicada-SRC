% Created 2023-11-20 lun 11:43
% Intended LaTeX compiler: pdflatex
\documentclass[11pt]{article}
\usepackage[utf8]{inputenc}
\usepackage[T1]{fontenc}
\usepackage{graphicx}
\usepackage{grffile}
\usepackage{longtable}
\usepackage{wrapfig}
\usepackage{rotating}
\usepackage[normalem]{ulem}
\usepackage{amsmath}
\usepackage{textcomp}
\usepackage{amssymb}
\usepackage{capt-of}
\usepackage{hyperref}
\usepackage[spanish, ]{babel}
\usepackage[margin=0.5in]{geometry}
\usepackage[svgnames,x11names]{xcolor}
\hypersetup{linktoc = all, colorlinks = true, urlcolor = DodgerBlue4, citecolor = PaleGreen1, linkcolor = SpringGreen4}
\PassOptionsToPackage{hyphens}{url}
\usepackage{nacal}
\usepackage{framed}
\usepackage{listings}
\input{hansl.tex}
\lstnewenvironment{hansl-gretl}
{\lstset{language={hansl},basicstyle={\ttfamily\footnotesize},numbers,rame=single,breaklines=true}}
{}
\newcommand{\hansl}[1]{\lstset{language={hansl},basicstyle={\ttfamily\small}}\lstinline{#1}}
\lstset{backgroundcolor=\color{lightgray!20}, }
\author{Marcos Bujosa}
\date{\today}
\title{Lección 10}
\begin{document}

\maketitle
\tableofcontents

\clearpage

\section{Precio de casas unifamiliares (Modelo Lin-log)}
\label{sec:org93b157e}
\begin{center}
\begin{tabular}{ll}
Guión: & \href{https://github.com/mbujosab/Ectr/tree/master/Practicas/Gretl/scripts/Houses2.inp}{Houses2.inp}\\
\end{tabular}
\end{center}

\begin{description}
\item[{Objetivo}] \begin{itemize}
\item Ajustar un modelo que incorpore posibles no linealidades en el  comportamiento de los precios.
\item Eliminar variables no significativas de un modelo.
\item Comparar el ajuste de dos modelos.
\item Interpretar los coeficientes de un modelo lineal-logarítmico.
\end{itemize}

\item[{Los datos}] Son los datos del ejemplo de clase, junto con dos
variables adicionales: número de dormitorios (\texttt{bedrms}) y cuartos de
baño (\texttt{baths}).

\item Motivación: Probablemente la valoración en el mercado de un metro
cuadrado adicional de casa no es la misma para casas grandes (pues
un metro más añade poca utilidad) que en casas pequeñas.

Por ello, una forma funcional que recoja esta no linealidad en el
comportamiento de los precios parece sensata.

Así pues, vamos a incorporar los regresores en logaritmos.

\item[{Para empezar}] primero cargue los datos de la base de datos de
Gretl
\begin{verbatim}
open data4-1
\end{verbatim}
\end{description}

\subsection{Tareas}
\label{sec:org1c7b635}

\begin{description}
\item[{Actividad 1}] Transforme las variables \texttt{sqft}, \texttt{bedrms} y \texttt{baths}\}
en logaritmos y ajuste el siguiente modelo de regresión lin-log
\begin{displaymath}
  \VA{price}=
  \beta_1+
  \beta_2\VA{l\_sqft}  +
  \beta_3\VA{l\_bedrms}+
  \beta_4\VA{l\_baths} +
  \per.
\end{displaymath}
  Marque con el ratón las variables \texttt{sqft}, \texttt{bedrms} y \texttt{baths} y
``pinche'' en \textbf{\emph{\texttt{Añadir -> Logaritmos de las variables
  seleccionadas}}}.

Ajuste por MCO el modelo indicado.

\emph{O bien ejecute el código}
\begin{verbatim}
logs sqft bedrms baths
Modelo1 <- ols price 0 l_sqft l_bedrms l_baths
\end{verbatim}

\item[{Actividad 2}] Observe que el parámetro asociado a \texttt{l\_baths}\} no es
estadísticamente significativo.

Pruebe a omitir dicha variable para ver si el modelo mejora.
\begin{verbatim}
omit l_baths
\end{verbatim}

\item[{Actividad 3}] Observe que el parámetro asociado a \texttt{l\_bedrms},
tampoco es estadísticamente significativo.

Repita con esta variable los pasos de la actividad anterior.
\begin{verbatim}
omit l_bedrms
\end{verbatim}

\item[{Actividad 4}] Habrá podido observar que aunque el parámetro
asociado a \texttt{l\_bedrms} no era estadísticamente significativo al 10\%,
si lo era al 12\%.

Como además ninguno de los estadísticos de selección de modelos ha
mejorado al omitir esta variable vamos a optar por incluirla de
nuevo.
\begin{itemize}
\item En la ventana del último modelo estimado, ``pinche'' en
\textbf{\emph{\texttt{Contrastes -> Añadir variables}}} y seleccione \texttt{l\_bedrms}.

\item Compruebe que mejoran los 3 estadísticos de selección de
modelos.
\end{itemize}

\emph{O bien ejecute el código}
\begin{verbatim}
Modelo2 <- add l_bedrms
\end{verbatim}

\item[{Actividad 5}] En un modelo con variable dependiente en niveles, y
regresor en logaritmos, la interpretación del parámetro \(\beta\)
asociado es la siguiente:
\begin{center}
\begin{tabular}{l}
\emph{Un incremento de un uno por ciento (\(0.01\)) del regresor supone un incremento del nivel del regresando de \(\beta*0.01\).}\\
\end{tabular}
\end{center}
 Empleando el último modelo, compruebe que 10,65 pies cuadrados
adicionales en la casa más pequeña tienen el mismo precio esperado
que 30 pies cuadrados en la casa más grande de la muestra.

\begin{itemize}
\item Amplíe la muestra en dos observaciones (\textbf{\emph{\texttt{Datos -> Añadir
    observaciones}}} y añada 2).

\begin{itemize}
\item Para la casa número 15, replique los datos de la 14, pero con 1\% más de superficie.
\item Para la casa número 16, replique los datos de la 1, pero  con 1\% más de superficie.
\end{itemize}

Marque \texttt{l\_sqft}, \texttt{l\_bedrms} y \texttt{l\_baths} y en \textbf{\emph{\texttt{Datos -> Editar
    los valores}}} añada los siguientes valores

\begin{table}[htbp]
\caption{Logaritmo de los datos para las casas 15 y 16 (réplica de las casa 14 y 1 salvo por el incremento en un 1\% de la superficie)}
\centering
\begin{tabular}{rlll}
obs. & \texttt{l\_sqft} & \texttt{l\_bedrms} & \texttt{l\_baths}\\
\hline
15 & 8,0163179=log(3000*1.01) & 8,0163179=log(4) & 1,0986123=log(3)\\
16 & 6,9806804=log(1065*1.01) & 1,0986123=log(3) & 0,55961579=log(1.75)\\
\end{tabular}
\end{table}
\emph{O bien ejecute el código}
\begin{verbatim}
dataset addobs 2

genr l_sqft[15]  =log(3000*1.01)
genr l_bedrms[15]=log(4)
genr l_baths[15] =log(3)

genr l_sqft[16]  =log(1065*1.01)
genr l_bedrms[16]=log(3)
genr l_baths[16] =log(1.75)
\end{verbatim}

\item Ajuste la muestra a las observaciones 1 a 16 (\textbf{\emph{\texttt{Muestra ->
    Establecer rango}}} y establezca el rango de 1 a 16)

\item Re-estime el modelo finalmente elegido

\item En la ventana del modelo re-estimado observe la predicción de los
precios (\textbf{\emph{\texttt{Análisis -> Predicciones}}} y ajuste el \textbf{\emph{\texttt{Dominio de
    predicción}}} para que incluya toda la muestra (observaciones 1 a
15).

\item Verifique que el aumento de precio (\(2984,8\) dólares) entre la
casa 1 y la 16 es el mismo que entre la casa 14 y la 15, pese a
que a la casa 15 se le ha agregado el triple de superficie que a
la 16.

\emph{O bien ejecute el código}
\begin{verbatim}
smpl 1 16
fcast --static predP
print predP -o

matrix P = {predP}
scalar d1 = P[16,1]-P[1,1]
scalar d2 = P[15,1]-P[14,1]
\end{verbatim}
\end{itemize}
\end{description}


\vspace{10pt}
\noindent
\subsection{Código completo de la práctica}
\label{sec:org38b8978}
\vspace{10pt}
\lstinputlisting{scripts/Houses2.inp}
\clearpage
\section{Relaciones lineales en las variables}
\label{sec:org643e246}
\begin{center}
\begin{tabular}{ll}
Guión: & \href{https://github.com/mbujosab/Ectr/tree/master/Practicas/Gretl/scripts/POE2-4.inp}{POE2-4.inp}\\
\end{tabular}
\end{center}

\vspace{10pt}
\noindent
\textbf{Código completo de la práctica}
\vspace{10pt}
\lstinputlisting{scripts/POE2-4.inp}
\clearpage

\section{Precio de casas unifamiliares}
\label{sec:org04f9c21}
\begin{center}
\begin{tabular}{ll}
Guión: & \href{https://github.com/mbujosab/Ectr/tree/master/Practicas/Gretl/scripts/RamanathanEX6-1.inp}{RamanathanEX6-1.inp}\\
\end{tabular}
\end{center}

\vspace{10pt}
\noindent
\textbf{Código completo de la práctica}
\vspace{10pt}
\lstinputlisting{scripts/RamanathanEX6-1.inp}

\clearpage

\section{Modelo para los salarios}
\label{sec:org9495088}
\begin{center}
\begin{tabular}{ll}
Guión: & \href{https://github.com/mbujosab/Ectr/tree/master/Practicas/Gretl/scripts/RamanathanEX6-5.inp}{RamanathanEX6-5.inp}\\
\end{tabular}
\end{center}

\vspace{10pt}
\noindent
\textbf{Código completo de la práctica}
\vspace{10pt}
\lstinputlisting{scripts/RamanathanEX6-5.inp}
\clearpage

\section{Otro modelo para los salarios}
\label{sec:org6971184}
\begin{center}
\begin{tabular}{ll}
Guión: & \href{https://github.com/mbujosab/Ectr/tree/master/Practicas/Gretl/scripts/RamanathanEX6-6.inp}{RamanathanEX6-6.inp}\\
\end{tabular}
\end{center}

\vspace{10pt}
\noindent
\textbf{Código completo de la práctica}
\vspace{10pt}
\lstinputlisting{scripts/RamanathanEX6-6.inp}
\clearpage

\section{Elasticidades en la demanda del transporte en autobús}
\label{sec:orgd0e7044}
\begin{center}
\begin{tabular}{ll}
Guión: & \href{https://github.com/mbujosab/Ectr/tree/master/Practicas/Gretl/scripts/RamanathanAp6-11.inp}{RamanathanAp6-11.inp}\\
\end{tabular}
\end{center}

\vspace{10pt}
\noindent
\textbf{Código completo de la práctica}
\vspace{10pt}
\lstinputlisting{scripts/RamanathanAp6-11.inp}
\clearpage
\end{document}
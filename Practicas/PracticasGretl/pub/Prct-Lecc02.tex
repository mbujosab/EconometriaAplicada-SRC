% Created 2024-09-17 mar 13:14
% Intended LaTeX compiler: pdflatex
\documentclass[11pt]{article}
\usepackage[utf8]{inputenc}
\usepackage[T1]{fontenc}
\usepackage{graphicx}
\usepackage{longtable}
\usepackage{wrapfig}
\usepackage{rotating}
\usepackage[normalem]{ulem}
\usepackage{amsmath}
\usepackage{amssymb}
\usepackage{capt-of}
\usepackage{hyperref}
\usepackage[spanish, ]{babel}
\usepackage[margin=0.5in]{geometry}
\usepackage[svgnames,x11names]{xcolor}
\hypersetup{linktoc = all, colorlinks = true, urlcolor = DodgerBlue4, citecolor = PaleGreen1, linkcolor = SpringGreen4}
\PassOptionsToPackage{hyphens}{url}
\usepackage{nacal}
\usepackage{framed}
\usepackage{listings}
\lstset{language = hansl, %
        showstringspaces = false, % 
        basicstyle = \footnotesize\ttfamily,
        keywordstyle = \color{red!50!black}\ttfamily,
        keywordstyle = [2]\color{green!40!black}\bfseries,
        keywordstyle = [3]\color{cyan!65!black}\ttfamily,
        keywordstyle = [4]\color{green!55!black}\ttfamily,
        commentstyle = \color{blue!70!black}\sffamily,       % 
        stringstyle = \color{magenta!75!black}\ttfamily,
        % backgroundcolor=\color{lightgray},
        % These commands partly sorted by meaning:
        morekeywords = {list, bundle, %
                matrices, strings, bundles, lists, %
                funcerr, hfplot, launch, midasreg, %
                pkg, plot, setopt, % 
                deriv, params, %
                for, foreach, while, 
                },
        morekeywords = [2]{array, defarray, null, %
                BFGSmin, BFGScmax, BFGScmin, %
                bread, bwrite, cdummify, cnumber, %
                cnameget, cnameset, dropcoll, %
                easterday, ecdf, exists, fevd, genseries, %
                getinfo, getkeys, GSSmax, GSSmin, %
                hfdiff, hfldiff, hflags, hflist, instring, %
                isdiscrete, isdummy, isoconv, isodate, %
                jsonget, jsongetb, juldate, kdsmooth, %
                kmeier, kpsscrit, ksetup, linearize, %
                lrcovar, mgradient, mlincomb, mweights, %
                naalen, nlines, NMmax, NMmin, %
                normtest, npcorr, NRmin, numhess, %
                pexpand, pxnobs, qlrpval, %             
                rnameget, rnameset, %
                seasonals, sleep, smplspan, sprintf, %
                square, stringify, strvals, substr, %
                typeof, varnames, xmlget, series},
        % the following for the double-dash options:
        %% moredelim       = [l][\color{green!50!gray}\ttfamily]{--},
        % obsolete former functions:
        deletekeywords = [2]{isnull, rowname, rownames, %
                colname, colnames, xpx },
        % and obsolete former commands:
        deletekeywords = {kalman, shell, sscanf},
        % try to redefine / override the $-accessors (not as group-1-keywords)
        keywordsprefix = {\$\$}, % anything which isn't a single $
        morekeywords = [3]{\$coeff,\$uhat, % to be completed
                \$ahat, \$aic, \$bic, \$chisq,
                \$command, \$compan, \$datatype, \$depvar, \$df,
                \$diagpval, \$diagtest, \$dw, \$dwpval, \$ec,
                \$error, \$ess , \$evals, \$fcast, \$fcse,
                \$fevd, \$Fstat, \$gmmcrit, \$h, \$hausman,
                \$hqc, \$huge, \$jalpha, \$jbeta, \$jvbeta,
                \$lang, \$llt, \$lnl, \$macheps, \$mnlprobs,
                \$model, \$ncoeff, \$nobs, \$nvars, \$obsdate,
                \$obsmajor, \$obsmicro, \$obsminor, \$pd , \$pi,
                \$pvalue, \$qlrbreak, \$rho, \$rsq, \$sample,
                \$sargan, \$sigma, \$stderr, \$stopwatch, \$sysA,
                \$sysB, \$sysGamma, \$sysinfo, \$system, \$T,
                \$t1, \$t2, \$test, \$tmax, \$trsq,
                \$uhat, \$unit, \$vcv, \$vecGamma, \$version,
                \$vma, \$windows, \$xlist, \$xtxinv, \$yhat,
                \$ylist},
        morekeywords = [4]{scalar}, %
        %morekeywords = [5]{randgen,mean,std}, %
    inputencoding = utf8,  % Input encoding
    extendedchars = true,  % Extended ASCII
    texcl         = true,  % Activate LaTeX commands in comments
    literate      =        % Support additional characters
      {á}{{\'a}}1  {é}{{\'e}}1  {í}{{\'i}}1 {ó}{{\'o}}1  {ú}{{\'u}}1
      {Á}{{\'A}}1  {É}{{\'E}}1  {Í}{{\'I}}1 {Ó}{{\'O}}1  {Ú}{{\'U}}1
      {ü}{{\"u}}1  {Ü}{{\"U}}1  {ñ}{{\~n}}1 {Ñ}{{\~N}}1  
}

\lstnewenvironment{hansl-gretl}
{\lstset{language={hansl},basicstyle={\ttfamily\footnotesize},numbers,rame=single,breaklines=true}}
{}
\newcommand{\hansl}[1]{\lstset{language={hansl},basicstyle={\ttfamily\small}}\lstinline{#1}}
\lstset{backgroundcolor=\color{lightgray!20}, }
\author{Marcos Bujosa}
\date{\today}
\title{Lección 2}
\begin{document}

\maketitle
\tableofcontents



\section{Componentes no observables del número de viajeros internacionales}
\label{sec:org6f2bd9e}
\begin{center}
\begin{tabular}{ll}
Guión: & \href{https://github.com/mbujosab/EconometriaAplicada-SRC/blob/main/Practicas/PracticasGretl/pub/scripts/componentesAirlinePass.inp}{componentesAirlinePass.inp}\\[0pt]
\end{tabular}
\end{center}

Vamos a reproducir el ejemplo de estimación de componentes no
observables visto en clase. Los datos son mensuales y contienen el
número de viajeros internacionales (en miles) en las líneas aéreas
norteamericanas en los años 1949--1960 que aparece en manual de Box \&
Jenkins.

\begin{enumerate}
\item Objetivo
\label{sec:org733ba9a}
\begin{enumerate}
\item Reproducir el ejemplo visto en la lección 2.
\end{enumerate}

\item Carga de datos
\label{sec:orga19a3b2}
\textbf{\emph{\texttt{Archivo -{}-{}> Abrir datos -{}-{}> Archivo de muestra}}} y en la pestaña
\textbf{\emph{\texttt{Gretl}}} seleccione \texttt{bjg}.

{\vspace{0pt} \footnotesize \color{gray!70!black}
\emph{o bien teclee en linea de comandos}:
\begin{verbatim}
open bjg
\end{verbatim}
}
\end{enumerate}

\subsection{Actividad 1 - generar una serie con el índice temporal y su cuadrado}
\label{sec:org9500fa3}
Pulse en el menú desplegable \textbf{\emph{\texttt{Añadir}}} que aparece arriba, en el centro de la
ventana principal de \href{https://gretl.sourceforge.net/es.html}{Gretl}.
\begin{itemize}
\item \textbf{\emph{\texttt{Añadir -> Variable índices}}}

{\vspace{1pt} \footnotesize \color{gray!70!black}
\emph{o bien teclee en linea de comandos}:
\begin{verbatim}
genr time
\end{verbatim}
}
\end{itemize}

Seleccione con el ratón la variable \texttt{time} y luego pulse en el menú desplegable \textbf{\emph{\texttt{Añadir}}} que aparece arriba, en el centro de la
ventana principal de \href{https://gretl.sourceforge.net/es.html}{Gretl}.
\begin{itemize}
\item \textbf{\emph{\texttt{Añadir -> Cuadrados de las variables seleccionadas}}}

{\vspace{0pt} \footnotesize \color{gray!70!black}
\emph{o bien teclee en linea de comandos}: 
\begin{verbatim}
square time
\end{verbatim}
}
\end{itemize}

\vspace{-3pt}

\subsection{Actividad 2 - Ajustar una tendencia lineal}
\label{sec:orga5aaf26}
\begin{itemize}
\item Estime el modelo mediante los menús desplegables: \textbf{\emph{\texttt{Modelo ->
  Mínimos Cuadrados Ordinarios}}}; indique a \href{https://gretl.sourceforge.net/es.html}{Gretl} el regresando y los
regresores y pulse \textbf{\emph{\texttt{Aceptar}}}.

\item ``Pinche'' \textbf{\emph{\texttt{Archivo}}} en la ventana del modelo estimado y
seleccione \textbf{\emph{\texttt{guardar como un icono y cerrar}}}

{\vspace{1pt} \footnotesize \color{gray!70!black} \color{gray!70!black}
\emph{o bien teclee en linea de comandos}:
\begin{verbatim}
TendenciaLineal  <- ols lg 0 time
\end{verbatim}
(\emph{el cero} \texttt{0} \emph{indica el término constante}: \texttt{const})
}

\item Recupere el modelo ``pinchando'' sobre su icono

{\vspace{1pt} \footnotesize \color{gray!70!black} \color{gray!70!black}
\emph{o teclee en linea de comandos el nombre que ha dado al icono
seguido de} \texttt{.show}, \emph{es decir}:
\begin{verbatim}
TendenciaLineal.show
\end{verbatim}
}
\end{itemize}

\begin{enumerate}
\item Recuperemos los valores ajustados
\label{sec:orgcd65ad1}
\begin{itemize}
\item Desde la ventana del modelo ajustado (recupérese con su icono),
``pinche'' en \textbf{\emph{\texttt{guardar -> valores estimados}}}. Elija como nombre
\texttt{phat} (puede añadir una descripción de la variable). Pulse en
\textbf{\emph{\texttt{Aceptar}}}
\item Repita para guardar los \texttt{residuos} con el nombre \texttt{ehat}

{\vspace{1pt} \footnotesize \color{gray!70!black}
\emph{o bien teclee en linea de comandos}:
\begin{verbatim}
series TendenciaLineal =  $yhat
series Comp_irregular =  $uhat
\end{verbatim}
}
\end{itemize}


\item Gráfico de la serie y la tendencia lineal
\label{sec:org74dc6b3}
\begin{itemize}
\item Marque las variables \texttt{lg} y \texttt{Tendencia} (pulsando \texttt{ctrl}). Pinche
con el botón derecho del ratón sobre ellas. Elija \textbf{\emph{\texttt{Gráfico de
  series temporales}}}

{\vspace{1pt} \footnotesize \color{gray!70!black}
\emph{o bien teclee en linea de comandos}:
\begin{verbatim}
GraficoTendenciaLineal <- gnuplot lg TendenciaLineal --time-series --with-lines
GraficoTendenciaLineal.show
\end{verbatim}
}
\end{itemize}

En la zona inferior izquierda de la ventana principal puede ver una
serie de iconos. Uno de ellos es la \textbf{\emph{\texttt{vista de iconos de sesión}}}.


\item Gráfico del componente irregular
\label{sec:orga31ea9b}

\begin{itemize}
\item Marque la variable correspondiente al componente irregular (pulsando
\texttt{ctrl}) y pinche con el botón derecho del ratón sobre ella. Elija
\textbf{\emph{\texttt{Gráfico de series temporales}}}

{\vspace{1pt} \footnotesize \color{gray!70!black}
\emph{o bien teclee en linea de comandos}:
\begin{verbatim}
GraficoComponenteIrregular <- gnuplot Comp_irregular --time-series --with-lines
GraficoComponenteIrregular.show
\end{verbatim}
}
\end{itemize}

\item Gráfico de la diferencia estacional del componente irregular
\label{sec:orgcb072b9}

\begin{itemize}
\item Seleccione con el ratón la variable correspondiente al componente
irregular y luego pulse en el menú desplegable \textbf{\emph{\texttt{Añadir}}} que
aparece arriba, en el centro de la ventana principal de \href{https://gretl.sourceforge.net/es.html}{Gretl}.
\begin{itemize}
\item \textbf{\emph{\texttt{Añadir -> Diferencias estacionales de las variables
      seleccionadas}}}
\end{itemize}

{\vspace{0pt} \footnotesize \color{gray!70!black}
\emph{o bien teclee en linea de comandos}: 
\begin{verbatim}
sdiff Comp_irregular
\end{verbatim}
}

\item Genere un gráfico con la nueva serie diferenciada estacionalmente

{\vspace{0pt} \footnotesize \color{gray!70!black}
\emph{en linea de comandos}: 
\begin{verbatim}
GraficoComponenteIrregularD12 <- gnuplot sd_Comp_irregular --time-series --with-lines
GraficoComponenteIrregularD12.show
\end{verbatim}
}
\end{itemize}
\end{enumerate}


\subsection{Actividad 3 - Ajustar una tendencia cuadrática}
\label{sec:org9173245}

Repita el ejercicio anterior, pero ajustando una tendencia cuadrática

\begin{itemize}
\item Estime la tendencia por MCO y vea los resultados de la regresión
\item Guarde los valores ajustados (TendenciaCuadratica)
\item Guarde los residuos (ComponenteIrregular2)
\item Dibuje la tendencia
\item Dibuje el componente irregular
\item Dibuje la diferencia estacional del componente irregular
\end{itemize}

{\vspace{1pt} \footnotesize \color{gray!70!black}
}


\subsection{Actividad 4 - Ajustar una tendencia cuadrática y un componente estacional determinista}
\label{sec:org2b46e04}

Pulse en el menú desplegable \textbf{\emph{\texttt{Añadir}}} que aparece arriba, en el centro de la
ventana principal de \href{https://gretl.sourceforge.net/es.html}{Gretl}.
\begin{itemize}
\item \textbf{\emph{\texttt{Añadir -> Variables ficticias estacionales}}}
\end{itemize}

{\vspace{1pt} \footnotesize \color{gray!70!black}
\emph{o bien teclee en linea de comandos}:
\begin{verbatim}
seasonals()
\end{verbatim}
}

\begin{itemize}
\item Estime el modelo mediante los menús desplegables: \textbf{\emph{\texttt{Modelo ->
  Mínimos Cuadrados Ordinarios}}}; indique a \href{https://gretl.sourceforge.net/es.html}{Gretl} el regresando y los
regresores y pulse \textbf{\emph{\texttt{Aceptar}}}.

\item ``Pinche'' \textbf{\emph{\texttt{Archivo}}} en la ventana del modelo estimado y
seleccione \textbf{\emph{\texttt{guardar como un icono y cerrar}}}

{\vspace{1pt} \footnotesize \color{gray!70!black} \color{gray!70!black}
\emph{o bien teclee en linea de comandos}:
\begin{verbatim}
ModeloCompleto  <- ols lg const time sq_time S1 S2 S3 S4 S5 S6 S7 S8 S9 S10 S11
ModeloCompleto.show
\end{verbatim}
}
\end{itemize}

\begin{enumerate}
\item Genere una nueva serie con la tendencia y otra con el componente estacional estimados
\label{sec:orgcd802ab}
\begin{itemize}
\item Cálculo de la tendencia estimada: \Estmc{\beta_1} \texttt{const} + \Estmc{\beta_2} \texttt{time} + \Estmc{\beta_3} \texttt{sq\_time}

\textbf{\emph{\texttt{Guardar -> Definir una nueva variable}}} y teclee:
{\vspace{1pt} \footnotesize \color{gray!70!black} \color{gray!70!black}
\begin{verbatim}
series Tendencia3 = $coeff[1] + $coeff[2]*time + $coeff[3]*sq_time
\end{verbatim}
}
o bien:
{\vspace{1pt} \footnotesize \color{gray!70!black} \color{gray!70!black}
\begin{verbatim}
series Tendencia3 = $coeff(const) + $coeff(time)*time + $coeff(sq_time)*sq_time
\end{verbatim}
}

\item De manera análoga genere una serie con el componente estacional

{\vspace{1pt} \footnotesize \color{gray!70!black} \color{gray!70!black}
\begin{verbatim}
series Comp_Estacional3 = $coeff(S1)*S1 + $coeff(S2)*S2 + $coeff(S3)*S3 + $coeff(S4)*S4 \
                        + $coeff(S5)*S5 + $coeff(S6)*S6 + $coeff(S7)*S7 + $coeff(S8)*S8 \
                        + $coeff(S9)*S9 + $coeff(S10)*S10 + $coeff(S11)*S11 
\end{verbatim}
}

\item Genere los siguientes gráficos
\begin{itemize}
\item la serie y su tendencia cuadrática

\item El componente estacional

\item El componente irregular

\item La serie y su ajuste
\end{itemize}

{\vspace{1pt} \footnotesize \color{gray!70!black} \color{gray!70!black}
\begin{verbatim}
GraficoTendencia3 <- gnuplot lg Tendencia3 --time-series --with-lines
GraficoTendencia3.show

GraficoComponenteEstacional3 <- gnuplot Comp_Estacional3 --time-series --with-lines
GraficoComponenteEstacional3.show

series ComponenteIrregular3 = $uhat
GraficoComponenteIrregular3 <- gnuplot ComponenteIrregular3 --time-series --with-lines
GraficoComponenteIrregular3.show

series Ajuste3 = $yhat
GraficoAjuste3 <- gnuplot lg Ajuste3 --time-series --with-lines
GraficoAjuste3.show
\end{verbatim}
}
\end{itemize}


\vspace{10pt}
\noindent
\textbf{Código completo de la práctica} \texttt{componentesAirlinePass.inp}
\vspace{10pt}
\lstinputlisting{scripts/componentesAirlinePass.inp}
\clearpage
\end{enumerate}


\section{Componentes no observables del número de viajeros internacionales 2}
\label{sec:org67d71e0}
\begin{center}
\begin{tabular}{ll}
Guión: & \href{https://github.com/mbujosab/EconometriaAplicada-SRC/blob/main/Practicas/PracticasGretl/pub/scripts/componentesAirlinePass2.inp}{componentesAirlinePass2.inp}\\[0pt]
\end{tabular}
\end{center}

Continuamos el ejemplo anterior, pero ahora vamos a reducir el modelo
quitando variables no significativas.

\subsection{Actividad 1 - Estime el modelo con tendencia cuadrática y estacionalidad determinista}
\label{sec:org530e692}

Repita la estimación del último modelo de la practica anterior

{\vspace{0pt} \footnotesize \color{gray!70!black}
\begin{verbatim}
open bjg
genr time
square time
seasonals()
ModeloInicial  <- ols lg const time sq_time S1 S2 S3 S4 S5 S6 S7 S8 S9 S10 S11
ModeloInicial.show
\end{verbatim}
}

\subsection{Actividad 2 - Reducir el modelo eliminando secuencialmente variables no significativas}
\label{sec:org3646755}

Los p-valores de algunos parámetros indican que sus estimaciones son
no significativas. En particular los correspondientes a las variables
ficticias de enero, febrero y octubre.

Reduzca el modelo, eliminando aquellas variables no significativas al
5\%. Verifique que el conjunto de variables excluidas es conjuntamente
no significativo.

\begin{itemize}
\item Desde la ventana del modelo estimado ``pinche'' en \textbf{\emph{\texttt{contrastes -{}-{}>
  omitir variables}}} y marque la opción \emph{\texttt{Eliminación secuencial de
  variables utilizando el valor p a dos colas}}, indique una
significación del 5\% y pulse en \textbf{\emph{\texttt{Aceptar}}}
{\vspace{0pt} \footnotesize \color{gray!70!black}
\begin{verbatim}
PrimeraReduccion <- omit --auto=0.05
\end{verbatim}
}
\end{itemize}


\subsection{Actividad 3 - Contrastar la ausencia de autocorrelación}
\label{sec:org027957e}

\begin{itemize}
\item Observe bajo el valor de contraste de Durbin-Watson (0,691477).

\item Desde la ventana del modelo estimado ``pinche'' en \textbf{\emph{\texttt{contrastes -{}-{}>
  valor p del estadístico Durbin-Watson}}}. 
{\vspace{0pt} \footnotesize \color{gray!70!black}
\emph{o bien teclee en linea de comandos}:
\begin{verbatim}
scalar DW = $dw
scalar PDW = $dwpval
print DW
print PDW
\end{verbatim}
}

Claramente se rechaza la ausencia de autocorrelación de orden uno.

\item Desde la ventana del modelo estimado ``pinche'' en \textbf{\emph{\texttt{contrastes -{}-{}>
  Autocorrelación}}}. E indique por ejemplo 3 retardos. 
{\vspace{0pt} \footnotesize \color{gray!70!black}
\emph{o bien teclee en linea de comandos}:
\begin{verbatim}
modtest --autocorr 3
\end{verbatim}
}

Claramente se rechaza la ausencia de autocorrelación y se observa
que el retardo de orden uno es muy significativo.
\end{itemize}


\subsection{Actividad 4 - Estimación del modelo con errores estándar robustos}
\label{sec:orge404a15}

Los test de autocorrelación indican que la inferencia empleada para
reducir el modelo es incorrecta. A la hora de calcular las
desviaciones típicas de las estimaciones hay que tener en cuenta que
las perturbaciones están autocorreladas.

\begin{itemize}
\item Estime el modelo inicial con errores estándar robustos: \textbf{\emph{\texttt{Modelo ->
  Mínimos Cuadrados Ordinarios}}}; indique a \href{https://gretl.sourceforge.net/es.html}{Gretl} el regresando y los
regresores; marque la opción \emph{\texttt{Desviaciones típicas robustas}} y
pulse \textbf{\emph{\texttt{Aceptar}}}.

{\vspace{1pt} \footnotesize \color{gray!70!black} \color{gray!70!black}
\emph{o bien teclee en linea de comandos}:
\begin{verbatim}
ModeloInicialDTR  <- ols lg const time sq_time S1 S2 S3 S4 S5 S6 S7 S8 S9 S10 S11 --robust
ModeloInicialDTR.show
\end{verbatim}
}

Fíjese que al 5\% de significación, solo es no significativa al
dummie correspondiente al mes de febrero.

\item Reduzca el modelo, eliminando aquellas variables no significativas
al 5\%. Verifique que el conjunto de variables excluidas es
conjuntamente no significativo.

Desde la ventana del modelo estimado ``pinche'' en \textbf{\emph{\texttt{contrastes -{}-{}>
  omitir variables}}} y marque la opción \emph{\texttt{Eliminación secuencial de
  variables utilizando el valor p a dos colas}}, indique una
significación del 5\% y pulse en \textbf{\emph{\texttt{Aceptar}}}
{\vspace{0pt} \footnotesize \color{gray!70!black}
\emph{o bien teclee en linea de comandos}:
\begin{verbatim}
ModeloReducidoDTR <- omit --auto=0.05
\end{verbatim}
}

Fíjese que únicamente se elimina la dummie correspondiente a
febrero.
\end{itemize}

\subsection{Actividad 4 - Estimación incluyendo en el modelo la autocorrelación de orden uno en las perturbaciones}
\label{sec:org63fba2b}

\begin{itemize}
\item Re-estime el modelo incluyendo en el modelo un AR(1) para las
perturbaciones: \textbf{\emph{\texttt{Modelo -> Series temporales univariantes ->
  Errores AR -> AR(1)}}} y pulse \textbf{\emph{\texttt{Aceptar}}}.

{\vspace{0pt} \footnotesize \color{gray!70!black}
\emph{o bien teclee en linea de comandos}:
\begin{verbatim}
ModeloAR1  <- ar1 lg const time sq_time S1 S2 S3 S4 S5 S6 S7 S8 S9 S10 S11
ModeloAR1.show
\end{verbatim}
}

\item Elimine secuencialmente las variables no significativas al 5\%

\item Haga un gráfico de los residuos y observe que "son estacionarios"
(es decir, que tienen el aspecto de una realización de un proceso
estacionario)
\item Haga un gráfico de frecuencias de los residuos y observe que tiene
la forma campaniforme compatible con una distribución gaussiana.

Marque con el ratón la variable \texttt{Residuos} y pinchado en la serie
marcada con el botón derecho del ratón seleccione \textbf{\emph{\texttt{Distribución de
  frecuencias}}}.

{\vspace{3pt} \color{gray!70!black}
\emph{O bien mediante el comando}
\begin{verbatim}
freq Residuos --show-plot
\end{verbatim}
}
donde \texttt{-{}-{}show-plot} indica que se genere el gráfico en una
ventana. Observe que dicho comando también genera una ventana de texto
con la distribución de frecuencias relativa y acumulada.

\item Realice el contraste de normalidad de los residuos: Desde la ventana
del modelo estimado ``pinche'' en \textbf{\emph{\texttt{contrastes -{}-{}> Normalidad de
  los residuos}}} y marque la opción \emph{\texttt{Eliminación secuencial de
  variables utilizando el valor p a dos colas}}, indique una
significación del 5\% y pulse en \textbf{\emph{\texttt{Aceptar}}}

O bien desde la ventana principal: marque la variable \texttt{Residuos} y
``pinche'' en \textbf{\emph{\texttt{Variable -{}-{}> Contraste de Normalidad}}}

{\vspace{3pt} \color{gray!70!black}
\emph{O bien mediante el comando}
\begin{verbatim}
normtest Residuos --all
\end{verbatim}
}
\end{itemize}


\vspace{10pt}
\noindent
\textbf{Código completo de la práctica} \texttt{componentesAirlinePass2.inp}
\vspace{10pt}
\lstinputlisting{scripts/componentesAirlinePass2.inp}
\clearpage
\end{document}

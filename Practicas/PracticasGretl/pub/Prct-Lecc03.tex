% Created 2024-09-17 mar 15:33
% Intended LaTeX compiler: pdflatex
\documentclass[11pt]{article}
\usepackage[utf8]{inputenc}
\usepackage[T1]{fontenc}
\usepackage{graphicx}
\usepackage{longtable}
\usepackage{wrapfig}
\usepackage{rotating}
\usepackage[normalem]{ulem}
\usepackage{amsmath}
\usepackage{amssymb}
\usepackage{capt-of}
\usepackage{hyperref}
\usepackage[spanish, ]{babel}
\usepackage[margin=0.5in]{geometry}
\usepackage[svgnames,x11names]{xcolor}
\hypersetup{linktoc = all, colorlinks = true, urlcolor = DodgerBlue4, citecolor = PaleGreen1, linkcolor = SpringGreen4}
\PassOptionsToPackage{hyphens}{url}
\usepackage{nacal}
\usepackage{framed}
\usepackage{listings}
\lstset{language = hansl, %
        showstringspaces = false, % 
        basicstyle = \footnotesize\ttfamily,
        keywordstyle = \color{red!50!black}\ttfamily,
        keywordstyle = [2]\color{green!40!black}\bfseries,
        keywordstyle = [3]\color{cyan!65!black}\ttfamily,
        keywordstyle = [4]\color{green!55!black}\ttfamily,
        commentstyle = \color{blue!70!black}\sffamily,       % 
        stringstyle = \color{magenta!75!black}\ttfamily,
        % backgroundcolor=\color{lightgray},
        % These commands partly sorted by meaning:
        morekeywords = {list, bundle, %
                matrices, strings, bundles, lists, %
                funcerr, hfplot, launch, midasreg, %
                pkg, plot, setopt, % 
                deriv, params, %
                for, foreach, while, 
                },
        morekeywords = [2]{array, defarray, null, %
                BFGSmin, BFGScmax, BFGScmin, %
                bread, bwrite, cdummify, cnumber, %
                cnameget, cnameset, dropcoll, %
                easterday, ecdf, exists, fevd, genseries, %
                getinfo, getkeys, GSSmax, GSSmin, %
                hfdiff, hfldiff, hflags, hflist, instring, %
                isdiscrete, isdummy, isoconv, isodate, %
                jsonget, jsongetb, juldate, kdsmooth, %
                kmeier, kpsscrit, ksetup, linearize, %
                lrcovar, mgradient, mlincomb, mweights, %
                naalen, nlines, NMmax, NMmin, %
                normtest, npcorr, NRmin, numhess, %
                pexpand, pxnobs, qlrpval, %             
                rnameget, rnameset, %
                seasonals, sleep, smplspan, sprintf, %
                square, stringify, strvals, substr, %
                typeof, varnames, xmlget, series},
        % the following for the double-dash options:
        %% moredelim       = [l][\color{green!50!gray}\ttfamily]{--},
        % obsolete former functions:
        deletekeywords = [2]{isnull, rowname, rownames, %
                colname, colnames, xpx },
        % and obsolete former commands:
        deletekeywords = {kalman, shell, sscanf},
        % try to redefine / override the $-accessors (not as group-1-keywords)
        keywordsprefix = {\$\$}, % anything which isn't a single $
        morekeywords = [3]{\$coeff,\$uhat, % to be completed
                \$ahat, \$aic, \$bic, \$chisq,
                \$command, \$compan, \$datatype, \$depvar, \$df,
                \$diagpval, \$diagtest, \$dw, \$dwpval, \$ec,
                \$error, \$ess , \$evals, \$fcast, \$fcse,
                \$fevd, \$Fstat, \$gmmcrit, \$h, \$hausman,
                \$hqc, \$huge, \$jalpha, \$jbeta, \$jvbeta,
                \$lang, \$llt, \$lnl, \$macheps, \$mnlprobs,
                \$model, \$ncoeff, \$nobs, \$nvars, \$obsdate,
                \$obsmajor, \$obsmicro, \$obsminor, \$pd , \$pi,
                \$pvalue, \$qlrbreak, \$rho, \$rsq, \$sample,
                \$sargan, \$sigma, \$stderr, \$stopwatch, \$sysA,
                \$sysB, \$sysGamma, \$sysinfo, \$system, \$T,
                \$t1, \$t2, \$test, \$tmax, \$trsq,
                \$uhat, \$unit, \$vcv, \$vecGamma, \$version,
                \$vma, \$windows, \$xlist, \$xtxinv, \$yhat,
                \$ylist},
        morekeywords = [4]{scalar}, %
        %morekeywords = [5]{randgen,mean,std}, %
    inputencoding = utf8,  % Input encoding
    extendedchars = true,  % Extended ASCII
    texcl         = true,  % Activate LaTeX commands in comments
    literate      =        % Support additional characters
      {á}{{\'a}}1  {é}{{\'e}}1  {í}{{\'i}}1 {ó}{{\'o}}1  {ú}{{\'u}}1
      {Á}{{\'A}}1  {É}{{\'E}}1  {Í}{{\'I}}1 {Ó}{{\'O}}1  {Ú}{{\'U}}1
      {ü}{{\"u}}1  {Ü}{{\"U}}1  {ñ}{{\~n}}1 {Ñ}{{\~N}}1  
}

\lstnewenvironment{hansl-gretl}
{\lstset{language={hansl},basicstyle={\ttfamily\footnotesize},numbers,rame=single,breaklines=true}}
{}
\newcommand{\hansl}[1]{\lstset{language={hansl},basicstyle={\ttfamily\small}}\lstinline{#1}}
\lstset{backgroundcolor=\color{lightgray!20}, }
\author{Marcos Bujosa}
\date{\today}
\title{Lección 3}
\begin{document}

\maketitle
\tableofcontents



\section{Correlación espuria entre la incidencia de melanoma en el estado de Connecticut y el PNB de EEUU}
\label{sec:orgf6c5b1e}
\begin{center}
\begin{tabular}{ll}
Guión: & \href{https://github.com/mbujosab/EconometriaAplicada-SRC/blob/main/Practicas/PracticasGretl/pub/scripts/GNPvsMelanoma.inp}{GNPvsMelanoma.inp}\\[0pt]
\end{tabular}
\end{center}

\begin{enumerate}
\item Objetivo
\label{sec:org3384657}
Comprobar cómo la aparente relación entre ambas series temporales se
desvanece al tomar primeras diferencias

\item Carga de datos
\label{sec:orgf5c07f6}
\textbf{\emph{\texttt{Archivo -{}-{}> Abrir datos -{}-{}> Archivo de usuario}}} y buscar en el
disco el fichero \texttt{GNPvsMelanoma.gdt}.

{\vspace{0pt} \footnotesize \color{gray!70!black}
\emph{o bien teclee en linea de comandos}:
\begin{verbatim}
open datos/GNPvsMelanoma.gdt
\end{verbatim}
}
\end{enumerate}

\subsection{Actividad 1 - Dibujar ambas series en un mismo gráfico}
\label{sec:orgf63e76f}
Marque las series \texttt{GNP} y \texttt{Melanoma}. Pulse sobre ellas con el botón
derecho del ratón. En el menú desplegable seleccione \textbf{\emph{\texttt{Gráfico de
Series Temporales}}} (indique representar en un único gráfico).

Guarde el gráfico en la sesión como un icono.

{\vspace{1pt} \footnotesize \color{gray!70!black}
\emph{o bien teclee en linea de comandos}:
\begin{verbatim}
GraficoSeriesEnNiveles <- gnuplot GNP Melanoma --time-series --with-lines
\end{verbatim}
}

\vspace{-3pt}

\subsection{Actividad 2 - Dibujar el diagrama de dispersión}
\label{sec:org7223072}
Marque las series \texttt{GNP} y \texttt{Melanoma}. Pulse sobre ellas con el botón
derecho del ratón. En el menú desplegable seleccione \textbf{\emph{\texttt{Gráfico de
dispersión XY}}} (elija como variable del eje X \texttt{Melanoma} y marque
suprimir la recta estimada).

Guarde el gráfico en la sesión como un icono.

{\vspace{1pt} \footnotesize \color{gray!70!black}
\emph{o bien teclee en linea de comandos}:
\begin{verbatim}
DiagramDispersion <- gnuplot GNP Melanoma --fit=none
\end{verbatim}
}

\vspace{-3pt}

\begin{enumerate}
\item ¿Tienen tendencia estas series temporales?
\item ¿Parece ser una tendencia común a ambas series?
\item Pero\ldots{} ¿lo podemos saber con seguridad solo mirando el gráfico?
\end{enumerate}

\subsection{Actividad 3 - Calcular la correlación entre ambas series}
\label{sec:orgc7c229d}
Marque las series \texttt{GNP} y \texttt{Melanoma}. Pulse sobre ellas con el botón
derecho del ratón. En el menú desplegable seleccione \textbf{\emph{\texttt{Matriz de
correlación}}}

{\vspace{0pt} \footnotesize \color{gray!70!black}
\emph{o bien teclee en linea de comandos}:
\begin{verbatim}
corr GNP Melanoma
\end{verbatim}
}

\subsection{Actividad 4 - Regresar \texttt{GNP} sobre \texttt{Melanoma} y constatar que el ajuste es bueno}
\label{sec:org37cd16b}
Estime el modelo mediante los menús desplegables: \textbf{\emph{\texttt{Modelo -> Mínimos
  Cuadrados Ordinarios}}}; indique a \href{https://gretl.sourceforge.net/es.html}{Gretl} el regresando y regresor y
  pulse \textbf{\emph{\texttt{Aceptar}}}.

{\vspace{0pt} \footnotesize \color{gray!70!black}
\emph{o bien teclee en linea de comandos}:
\begin{verbatim}
AjusteEnNiveles <- ols GNP 0 Melanoma
\end{verbatim}
}

Aunque el coeficiente de determinación es muy elevado y los parámetros
muy significativos, el modelo "\emph{no es válido}". Una forma de
constatarlo es darse cuenta de que si fuera cierto que

$$\boldsymbol{y}=\beta_1 \boldsymbol{1} + \beta_2 \boldsymbol{x} +
\boldsymbol{u}$$

Entonces también sería cierto que (y nótese que
\(\nabla\boldsymbol{1}=\boldsymbol{0}\))

$$\nabla\boldsymbol{y}=\beta_2 \nabla\boldsymbol{x} +
\nabla\boldsymbol{u}$$

Consecuentemente, si \(\boldsymbol{y}\) corresponde al \texttt{GNP} y
\(\boldsymbol{x}\) a \texttt{Melanoma}, al regresar la primera diferencia de
\texttt{GNP} sobre la primera diferencia de \texttt{Melanoma} el ajuste debería
indicar que el parámetro de la constante (\(\beta_1\)) no es
significativo, pero si la pendiente (\(\beta_2\)). Veamos que ocurre justo lo contrario\ldots{}

\subsection{Actividad 5 - Añadir la primera diferencia de los datos}
\label{sec:org25461a3}

Seleccione con el ratón la variable \texttt{GNP} y \texttt{Melanoma}. Luego pulse en el menú desplegable \textbf{\emph{\texttt{Añadir}}} que aparece arriba, en el centro de la
ventana principal de \href{https://gretl.sourceforge.net/es.html}{Gretl}.
\begin{itemize}
\item \textbf{\emph{\texttt{Añadir -> Primeras diferencias de las variables seleccionadas}}}
\end{itemize}

Haga un gráfico con ambas series (verá que la tendencia ha desaparecido y que ya no se parecen entre sí).

Calcule también la correlación entre ambas series diferenciadas
(recuerde que en un modelo lineal simple el cuadrado de dicha
correlación es el coeficiente de determinación).

{\vspace{0pt} \footnotesize \color{gray!70!black}
\emph{o bien teclee en linea de comandos}: 
\begin{verbatim}
diff GNP Melanoma
GraficoSeriesEnDiferencias <- gnuplot d_GNP d_Melanoma --time-series --with-lines
corr d_GNP d_Melanoma
\end{verbatim}
}

\subsection{Actividad 6 - Regresar \texttt{d\_GNP} sobre \texttt{d\_Melanoma} y constatar que el ajuste es horrible}
\label{sec:org5a0a75a}

Estime el modelo mediante los menús desplegables: \textbf{\emph{\texttt{Modelo -> Mínimos
  Cuadrados Ordinarios}}}; indique a \href{https://gretl.sourceforge.net/es.html}{Gretl} el regresando y regresor y
  pulse \textbf{\emph{\texttt{Aceptar}}}.

{\vspace{0pt} \footnotesize \color{gray!70!black}
\emph{o bien teclee en linea de comandos}:
\begin{verbatim}
AjusteEnDiferencias <- ols d_GNP 0 d_Melanoma
\end{verbatim}
}

\textbf{Conclusión.} Las variables \texttt{GNP} y \texttt{Melanoma} muestran una tendencia
creciente, lo que conduce a un elevado coeficiente de correlación
entre ellas; pero la tendencia ni es común, ni la correlación se puede
atribuir a ninguna relación de causalidad entre ellas. La correlación
en este caso es espuria (es decir, carece de sentido tratar de
interpretarla).

\vspace{10pt}
\noindent
\textbf{Código completo de la práctica} \texttt{GNPvsMelanoma.inp}
\vspace{10pt}
\lstinputlisting{scripts/GNPvsMelanoma.inp}
\clearpage


\section{Tipos de interés a corto y largo plazo}
\label{sec:org5f3b629}
\begin{center}
\begin{tabular}{ll}
Guión: & \href{https://github.com/mbujosab/EconometriaAplicada-SRC/blob/main/Practicas/PracticasGretl/pub/scripts/UKinterestRates.inp}{UKinterestRates.inp}\\[0pt]
\end{tabular}
\end{center}

\begin{enumerate}
\item Objetivo
\label{sec:org0710261}
Ver que probablemente la correlación entre los tipos a corto y largo
plazo no es espuria, y que ambas series probablemente están
cointegrados.

\item Carga de datos
\label{sec:org3e0aea3}
\textbf{\emph{\texttt{Archivo -{}-{}> Abrir datos -{}-{}> Archivo de usuario}}} y buscar en el
disco el fichero \texttt{UKinterestRates.gdt}.

{\vspace{0pt} \footnotesize \color{gray!70!black}
\emph{o bien teclee en linea de comandos}:
\begin{verbatim}
open datos/UKinterestRates.gdt
\end{verbatim}
}
\end{enumerate}

\subsection{Actividad 1 - Dibujar ambas series en un mismo gráfico}
\label{sec:org1705b77}
Marque las series \texttt{Long} y \texttt{Short}. Pulse sobre ellas con el botón
derecho del ratón. En el menú desplegable seleccione \textbf{\emph{\texttt{Gráfico de
Series Temporales}}} (indique representar en un único gráfico).

Guarde el gráfico en la sesión como un icono.

{\vspace{1pt} \footnotesize \color{gray!70!black}
\emph{o bien teclee en linea de comandos}:
\begin{verbatim}
GraficoSeriesEnNiveles <- gnuplot Long Short --time-series --with-lines
\end{verbatim}
}

\vspace{-3pt}

\begin{enumerate}
\item ¿Tienen tendencia estas series temporales?
\item ¿Parece ser una tendencia común a ambas series?
\item Pero\ldots{} ¿lo podemos saber con seguridad solo mirando el gráfico?
\end{enumerate}

\subsection{Actividad 2 - Dibujar el diagrama de dispersión y calcule la correlación}
\label{sec:org02f8cf9}
Marque las series \texttt{Long} y \texttt{Short}. Pulse sobre ellas con el botón
derecho del ratón. En el menú desplegable seleccione \textbf{\emph{\texttt{Gráfico de
dispersión XY}}} (elija como variable del eje X \texttt{Long} y marque
suprimir la recta estimada).

Guarde el gráfico en la sesión como un icono.

Calcule la correlación entre ambas series.

{\vspace{1pt} \footnotesize \color{gray!70!black}
\emph{o bien teclee en linea de comandos}:
\begin{verbatim}
DiagramDispersion <- gnuplot Short Long --fit=none
corr Long Short
\end{verbatim}
}

\vspace{-3pt}

\subsection{Actividad 3 - Regrese la primera diferencia de los tipos a corto sobre la diferencia de los tipos a largo}
\label{sec:orgab6164a}
\begin{enumerate}
\item Incluya las primeras diferencias de \texttt{Short} y \texttt{Long}
\item Dibuje ambas series diferenciadas. ¿Parecen "ser estacionarias en
media"? ¿Son \texttt{Short} y \texttt{Long} aparentemente \(I(1)?\)
\item ¿Están correladas?
\item Regrese \texttt{d\_Short} sobre \texttt{d\_Long}
\item Observe los resultados de la regresión.
\begin{itemize}
\item ¿Son significativos los parámetros? ¿cuales sí y cuales no?
(compare esto con lo que pasaba en el ejemplo anterior)
\item ¿Reproduce el modelo parte de la varianza de \texttt{d\_Short}?
\end{itemize}
\end{enumerate}

Realice los pasos con la interfaz gráfica y los menús desplegables,
{\vspace{0pt} \footnotesize \color{gray!70!black}
\emph{o bien teclee en linea de comandos}:
\begin{verbatim}
diff Short Long
GraficoSeriesEnDiferencias <- gnuplot d_Short d_Long --time-series --with-lines
corr d_Short d_Long
AjusteEnDiferencias <- ols d_Short 0 d_Long
\end{verbatim}
}

\subsection{Actividad 4 - Ajuste los tipos corto plazo en función de los tipos a largo}
\label{sec:org19a9202}

Veamos si las series en niveles pueden estar cointegradas. Para ello
debemos analizar los residuos de la regresión de \texttt{Short} sobre
\texttt{Long}.

\begin{enumerate}
\item Regrese \texttt{Short} sobre \texttt{Long}
\item Observe los resultados de la regresión.
\begin{itemize}
\item ¿Son significativos los parámetros?
\item ¿Reproduce el modelo parte de la varianza de \texttt{Short}?
\end{itemize}
\item Dibuje los residuos de la regresión. ¿Parecen "estacionarios en
media"? dicho de otra forma ¿muestran alguna tendencia?
\end{enumerate}

Realice los pasos con la interfaz gráfica y los menús desplegables,
{\vspace{0pt} \footnotesize \color{gray!70!black}
\emph{o bien teclee en linea de comandos}:
\begin{verbatim}
AjusteEnNiveles <- ols Short 0 Long
residuos = $uhat
GraficoResiduos <- gnuplot residuos --time-series --with-lines
\end{verbatim}
}


\textbf{Conclusión.} Las variables \texttt{Short} y \texttt{Long} son no estacionarias
(tienen tendencia), lo que conduce a un elevado coeficiente de
correlación entre ellas; sus primeras diferencias parecen
"estacionarias" lo que sugiere que ambas series son \(I(1)\).

La regresión de las series en diferencias y los residuos de la
regresión en niveles parecen compatibles con que \texttt{Short} y \texttt{Long}
estén cointegradas, es decir, que tengan una tendencia común.

En este caso la correlación no parece ser espuria, por lo que cabe
interpretar dicha correlación y pensar que hay relación entre los
tipos a corto y a largo plazo.

A este análisis le falta la realización de contrastes estadísticos que
confirmen la estacionariedad de las series diferenciadas y de los
residuos de la última regresión.

\vspace{10pt}
\noindent
\textbf{Código completo de la práctica} \texttt{UKinterestRates.inp}

\vspace{10pt}
\lstinputlisting{scripts/UKinterestRates.inp}
\clearpage
\end{document}

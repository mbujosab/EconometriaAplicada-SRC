% Created 2024-07-27 sáb 14:12
% Intended LaTeX compiler: pdflatex
\documentclass[11pt]{article}
\usepackage[utf8]{inputenc}
\usepackage{lmodern}
\usepackage[T1]{fontenc}
\usepackage[top=1in, bottom=1.in, left=1in, right=1in]{geometry}
\usepackage{graphicx}
\usepackage{longtable}
\usepackage{float}
\usepackage{wrapfig}
\usepackage{rotating}
\usepackage[normalem]{ulem}
\usepackage{amsmath}
\usepackage{textcomp}
\usepackage{marvosym}
\usepackage{wasysym}
\usepackage{amssymb}
\usepackage{amsmath}
\usepackage[theorems, skins]{tcolorbox}
\usepackage[version=3]{mhchem}
\usepackage[numbers,super,sort&compress]{natbib}
\usepackage{natmove}
\usepackage{url}
\usepackage[cache=false]{minted}
\usepackage[strings]{underscore}
\usepackage[linktocpage,pdfstartview=FitH,colorlinks,
linkcolor=blue,anchorcolor=blue,
citecolor=blue,filecolor=blue,menucolor=blue,urlcolor=blue]{hyperref}
\usepackage{attachfile}
\usepackage{setspace}
\usepackage{lmodern}
\usepackage{tabularx}
\usepackage{booktabs}
\author{Marcos Bujosa}
\date{\today}
\title{Notas de cómo hacer algunas cosas con R}
\begin{document}

\maketitle

\begin{ABSTRACT}
Algunas notas de cómo hacer cosas en R.
\end{ABSTRACT}
\subsubsection*{Carga de algunas librerías de R que vamos a usar aquí}
\label{sec:org804f088}
\begin{minted}[frame=lines,fontsize=\scriptsize,linenos=]{R}
library(readr)        # para leer ficheros CSV
library(zoo)          # para leer datos con índices temporales
library(ggplot2)
library(jtools)       # para representación resultados estimación (summ)
\end{minted}
y además fijamos los parámetros por defecto para las figuras en \texttt{png}
del notebook
\begin{minted}[frame=lines,fontsize=\scriptsize,linenos=]{R}
# fijamos el tamaño de las figuras que se generan en el notebook
options(repr.plot.width = 12, repr.plot.height = 4, repr.plot.res = 200)
\end{minted}
\section{De \texttt{CSV} a \texttt{data\_frame}}
\label{sec:org504daff}

\begin{minted}[frame=lines,fontsize=\scriptsize,linenos=]{R}
datos_df <- read_csv('datos/GNPvsMelanoma.csv',show_col_types = FALSE)
head(datos_df, 3)
\end{minted}

\phantomsection
\label{}
\begin{center}
\begin{tabular}{rrr}
obs & GNP & Melanoma\\
\hline
<dbl> & <dbl> & <dbl>\\
1936 & 193.0 & 1.0\\
1937 & 203.2 & 0.8\\
1938 & 192.9 & 0.8\\
\end{tabular}
\end{center}
\section{De \texttt{data\_frame} a \texttt{ts}}
\label{sec:orga2b3057}

Este método permite volver al \texttt{data\_frame}

\begin{minted}[frame=lines,fontsize=\scriptsize,linenos=]{R}
datos_ts   <- ts(data = datos_df,
                 start = 1936,
                 end = 1972,
                 frequency = 1)
head(datos_ts, 3)
\end{minted}

\phantomsection
\label{}
\begin{center}
\begin{tabular}{rrr}
obs & GNP & Melanoma\\
\hline
1936 & 193.0 & 1.0\\
1937 & 203.2 & 0.8\\
1938 & 192.9 & 0.8\\
\end{tabular}
\end{center}
\section{De \texttt{CSV} a \texttt{ts} (vía \texttt{zoo})}
\label{sec:org1eac034}

\subsection{Primero \texttt{zoo}}
\label{sec:org2c8be8b}

\begin{minted}[frame=lines,fontsize=\scriptsize,linenos=]{R}
data <- read.zoo('datos/GNPvsMelanoma.csv', header=TRUE, index.column = 1, sep=",", FUN = as.yearmon)
class(data)
head(data, 3)
plot(data)
plot(data[,'Melanoma'])
\end{minted}

\phantomsection
\label{}
'zoo'
\begin{verbatim}
           GNP Melanoma
ene 1936 193.0      1.0
ene 1937 203.2      0.8
ene 1938 192.9      0.8
\end{verbatim}

\begin{center}
\includegraphics[width=.9\linewidth]{./.ob-jupyter/830f097ae96cace5d8551bab02d80b0909167ec8.png}
\end{center}
\begin{center}
\includegraphics[width=.9\linewidth]{./.ob-jupyter/36ee1e6970faaee38291f3c39058cbbca9d90da2.png}
\end{center}
\subsection{Luego a \texttt{ts}}
\label{sec:org87ad3f1}

\begin{minted}[frame=lines,fontsize=\scriptsize,linenos=]{R}
data_ts = as.ts( data )
class(data_ts)
head(data_ts, 3)

plot(data_ts[,'GNP'])
\end{minted}

\phantomsection
\label{}
\begin{enumerate}
\item 'mts'
\item 'ts'
\item 'matrix'
\end{enumerate}
\begin{center}
\begin{tabular}{rr}
GNP & Melanoma\\
\hline
193.0 & 1.0\\
203.2 & 0.8\\
192.9 & 0.8\\
\end{tabular}
\end{center}
\begin{figure}[htbp]
\centering
\includegraphics[width=.9\linewidth]{./.ob-jupyter/2faba2bd21f48f6ca00f34a3db03dd81b8a2356d.png}
\caption{A matrix: 3 × 2 of type dbl}
\end{figure}
\section{De dataframe a ts}
\label{sec:org798f20b}


\begin{minted}[frame=lines,fontsize=\scriptsize,linenos=]{R}
#+BEGIN_SRC jupyter-R
tseries <- as.ts( read.zoo( datos_df ) )
head(tseries)
plot(tseries)
\end{minted}

\phantomsection
\label{}
\begin{center}
\begin{tabular}{rr}
GNP & Melanoma\\
\hline
193.0 & 1.0\\
203.2 & 0.8\\
192.9 & 0.8\\
209.4 & 1.4\\
227.2 & 1.2\\
263.7 & 1.0\\
\end{tabular}
\end{center}
\begin{figure}[htbp]
\centering
\includegraphics[width=.9\linewidth]{./.ob-jupyter/5754f919bdc2564f8e6d8f8ee0681f5fe000ed0c.png}
\caption{A matrix: 6 × 2 of type dbl}
\end{figure}
\section{De \texttt{ts} a \texttt{data\_frame}}
\label{sec:org47f0c9a}

\begin{minted}[frame=lines,fontsize=\scriptsize,linenos=]{R}
DF = data.frame(date = zoo::as.Date(time(tseries[,"GNP"])),
                GNP = as.matrix(tseries[,"GNP"]),
                Melanoma = as.matrix(tseries[,"Melanoma"]))
head(DF,3)
\end{minted}

\phantomsection
\label{}
\begin{center}
\begin{tabular}{rrrr}
 & date & GNP & Melanoma\\
\hline
 & <date> & <dbl> & <dbl>\\
1 & 1936-01-01 & 193.0 & 1.0\\
2 & 1937-01-01 & 203.2 & 0.8\\
3 & 1938-01-01 & 192.9 & 0.8\\
\end{tabular}
\end{center}
\section{Resumen: dos formas de pasar de datos anuales en \texttt{CSV} a \texttt{ts}}
\label{sec:org7884c9a}

Una forma
\begin{minted}[frame=lines,fontsize=\scriptsize,linenos=]{R}
time_series <- ts(data = read_csv('datos/GNPvsMelanoma.csv',show_col_types = FALSE),
                  start = 1936,
                  end = 1972,
                  frequency = 1)
head(time_series, 2)
\end{minted}

\phantomsection
\label{}
\begin{center}
\begin{tabular}{rrr}
obs & GNP & Melanoma\\
\hline
1936 & 193.0 & 1.0\\
1937 & 203.2 & 0.8\\
\end{tabular}
\end{center}

donde \texttt{read\_csv('datos/GNPvsMelanoma.csv',show\_col\_types = FALSE)} es un \texttt{data\_frame}


Otra forma (vía \texttt{zoo})
\begin{minted}[frame=lines,fontsize=\scriptsize,linenos=]{R}
# library(zoo)
data_ts = as.ts( read.zoo('datos/GNPvsMelanoma.csv', header=TRUE, index.column = 1, sep=",", FUN = as.yearmon) )
head(data_ts, 3)
\end{minted}

\phantomsection
\label{}
\begin{center}
\begin{tabular}{rr}
GNP & Melanoma\\
\hline
193.0 & 1.0\\
203.2 & 0.8\\
192.9 & 0.8\\
\end{tabular}
\end{center}
\section{Gráfico de dos series temporales con sendos ejes verticales}
\label{sec:org68b5d49}

\begin{minted}[frame=lines,fontsize=\scriptsize,linenos=]{R}
# Mostrando la serie GNP
p <- autoplot(as.zoo(data_ts[,'GNP']))
p <- p + geom_line(aes(y = as.zoo(data_ts[,'GNP'])), colour="blue")

# como tienen escalar distintas se requiere ajustar los datos
sf<-max(data_ts[,'GNP'])/max(data_ts[,'Melanoma'])

# Se agrega Melanoma a Y multiplicada por el factor
p <- p + geom_line(aes(y = as.zoo(data_ts[,'Melanoma'])*sf), colour="red")
p <- p + scale_y_continuous(sec.axis = sec_axis(~./sf, name = "Incidencia casos de melanoma"))
p <- p + labs(y = "GNP",
              x = "Fechas")

# Se modifican los colores de los ejes
p <- p + theme(
    axis.title.y.left=element_text(color="blue"),
    axis.text.y.left=element_text(color="blue"),
    axis.ticks.y.left = element_line(color = "blue"),
    axis.title.y.right=element_text(color="red"),
    axis.text.y.right=element_text(color="red"),
    axis.ticks.y.right = element_line(color = "red")
  )
p
\end{minted}

\phantomsection
\label{}
\begin{center}
\includegraphics[width=.9\linewidth]{./.ob-jupyter/b222b0f9346ec6bd7ee140e2bdaf4c330523f733.png}
\end{center}
\subsection{Desde el dataframe}
\label{sec:orgec9220e}

\begin{minted}[frame=lines,fontsize=\scriptsize,linenos=]{R}
# Mostrando la serie GNP
p <- ggplot(DF, aes(x = date))
p <- p + geom_line(aes(y = GNP), colour="blue")

# como tienen escalar distintas se requiere ajustar los datos
sf<-max(DF['GNP'])/max(DF['Melanoma'])

# Se agrega Melanoma a Y multiplicada por el factor
p <- p + geom_line(aes(y = Melanoma*sf), colour="red")

p <- p + scale_y_continuous(sec.axis = sec_axis(~./sf, name = "Casos de melanoma"))
p <- p + labs(y = "GNP",
              x = "Fechas")
# Se modifican los colores de los ejes
p <- p + theme(
    axis.title.y.left=element_text(color="blue"),
    axis.text.y.left=element_text(color="blue"),
    axis.ticks.y.left = element_line(color = "blue"),
    axis.title.y.right=element_text(color="red"),
    axis.text.y.right=element_text(color="red"),
    axis.ticks.y.right = element_line(color = "red")
  )
p
\end{minted}

\phantomsection
\label{}
\begin{center}
\includegraphics[width=.9\linewidth]{./.ob-jupyter/3be9906e34c3328f6897efbef175c3ed5e0e683b.png}
\end{center}
\subsubsection{Y otra manera con \texttt{xyplot} (de \texttt{latticeExtra})}
\label{sec:org868a8a5}

Y ahora generamos el gráfico
\begin{minted}[frame=lines,fontsize=\scriptsize,linenos=]{R}
library(latticeExtra) # alternativa para gráficos con doble eje vertical (doubleYScale)
kk <- xyplot(GNP + Melanoma ~ date, DF, type="l")
# se agrega dos ejes Y se construye cada serie por separado
obj1 <- xyplot(GNP ~ date, DF, type = "l" , lwd=2, ylab="GNP (miles de millones de $)",  xlab="Years")
obj2 <- xyplot(Melanoma ~ date, DF, type = "l", lwd=2, ylab="Casos de melanoma")
# --> se realiza la grafica con el segundo eje Y
doubleYScale(obj1, obj2, add.ylab2 = TRUE)
\end{minted}

\phantomsection
\label{}
\begin{center}
\includegraphics[width=.9\linewidth]{./.ob-jupyter/33e057f85f65c733c486361f74b08bc1e8d2c473.png}
\end{center}
\section{Añadir nueva columna a un \texttt{ts}}
\label{sec:org232f3cd}


\begin{minted}[frame=lines,fontsize=\scriptsize,linenos=]{R}
d_GNP      = diff(datos_ts[,"GNP"])
d_Melanoma = diff(datos_ts[,"Melanoma"])
\end{minted}

\phantomsection
\label{}
\begin{verbatim}
5f061e65-c149-4dee-bc7d-a5338e092fb4
\end{verbatim}


este método cambia el nombre de las primeras columnas

\begin{minted}[frame=lines,fontsize=\scriptsize,linenos=]{R}
serie = ts.union(data_ts, d_GNP, d_Melanoma)
head(serie,3)
\end{minted}

\phantomsection
\label{}
\begin{verbatim}
76659920-41c9-4bd6-a654-09e26c8b2a20
\end{verbatim}


Este método es mejor, pero es pesado
\begin{minted}[frame=lines,fontsize=\scriptsize,linenos=]{R}
serie = ts.union(GNP = data_ts[,'GNP'], Melanoma = data_ts[,'Melanoma'], d_GNP, d_Melanoma)
head(serie,3)
\end{minted}

\phantomsection
\label{}
\begin{verbatim}
db11562d-7198-478f-953f-40fdfaf12282
\end{verbatim}


\begin{minted}[frame=lines,fontsize=\scriptsize,linenos=]{R}
DF.diferencias = data.frame(date = zoo::as.Date(time(d_GNP)),
                            d_GNP = as.matrix(d_GNP),
                            d_Melanoma = as.matrix(d_Melanoma))
head(DF.diferencias, 2)
\end{minted}

\phantomsection
\label{}
\begin{verbatim}
cd42271e-0f3a-4cd9-a1c4-1cdf0b4cb9c9
\end{verbatim}
\section{Otros}
\label{sec:org49630d6}

\begin{minted}[frame=lines,fontsize=\scriptsize,linenos=]{R}
mal_modelo <- lm(d_GNP ~ d_Melanoma)
summ( mal_modelo )
\end{minted}

\phantomsection
\label{}
\begin{verbatim}
1ceafe33-19a8-41f8-885a-cec3987c19fa
\end{verbatim}


\begin{minted}[frame=lines,fontsize=\scriptsize,linenos=]{R}
plot(as.ts(resid(mal_modelo))) 
abline(0,0) 
\end{minted}

\phantomsection
\label{}
\begin{verbatim}
9fb7d7f2-2948-46c7-8240-ca2e9384cc2d
\end{verbatim}
\end{document}

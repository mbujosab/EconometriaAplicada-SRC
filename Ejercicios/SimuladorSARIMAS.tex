% Created 2024-10-17 jue 18:33
% Intended LaTeX compiler: pdflatex
\documentclass[11pt]{article}
\usepackage[utf8]{inputenc}
\usepackage{lmodern}
\usepackage[T1]{fontenc}
\usepackage[top=1in, bottom=1.in, left=1in, right=1in]{geometry}
\usepackage{graphicx}
\usepackage{longtable}
\usepackage{float}
\usepackage{wrapfig}
\usepackage{rotating}
\usepackage[normalem]{ulem}
\usepackage{amsmath}
\usepackage{textcomp}
\usepackage{marvosym}
\usepackage{wasysym}
\usepackage{amssymb}
\usepackage{amsmath}
\usepackage[theorems, skins]{tcolorbox}
\usepackage[version=3]{mhchem}
\usepackage[numbers,super,sort&compress]{natbib}
\usepackage{natmove}
\usepackage{url}
\usepackage[cache=false]{minted}
\usepackage[strings]{underscore}
\usepackage[linktocpage,pdfstartview=FitH,colorlinks,
linkcolor=blue,anchorcolor=blue,
citecolor=blue,filecolor=blue,menucolor=blue,urlcolor=blue]{hyperref}
\usepackage{attachfile}
\usepackage{setspace}
\usepackage[spanish, ]{babel}
\usepackage{lmodern}
\usepackage{tabularx}
\usepackage{booktabs}
\author{Marcos Bujosa}
\date{\today}
\title{Simulación de múltiples modelos lineales}
\begin{document}

\maketitle
\section{Objetivo}
\label{sec:orgcef27e7}

Que el estudiante tenga una gran batería de series para ejercitar la
identificación. Los nombres de los ficheros impiden saber a priori de
qué modelo se trata en cada caso (cada serie tiene un hash único). El
alumno tendrá que observar el gráfico \emph{rango-media} para decidir
primero si la serie necesita la transformación logarítmica. Después
tendrá que ir mirando el gráfico de las series, de los residuos, los
correlogramas (también es útil mirar el gráfico del espectro del
modelo junto al periodograma) para intentar indentificar el modelo
lineal con el que se ha simulado la serie temporal.

Los ficheros \texttt{Etiqueta.txt} almacenan el tipo de modelo que se ha
simulado en cada caso. Por tanto, una vez se ha identificado un
modelo, se puede verificar si efectivamente ese ha sido el modelo
usado al crear la serie buscando el hash del fichero en la primera
columna de \texttt{Etiqueta.txt}.
\section{Qué hace el código de más abajo}
\label{sec:org42f97c3}

Simula desde procesos de ruido blanco a procesos
ARIMA\((p,d,q)\times(P,D,Q)_S\), con transformaciones exponenciales en
algunos casos y con medias distintas de cero en otros.

En total se simulan 320 modelos con periodicidad trimestral y 320 con
periodicidad mensual.

Las series se simulan con la librería \texttt{tfarima} de R escrita por José
Luis Gallego y cada una de ellas se graba en un fichero \texttt{csv} con un
nombre anonimizado (mediante un hash) para ocultar cualquier pista
sobre cuál es el modelo subyacente.

Para mayor comodidad del estudiante, cada serie es leída
posteriormente con GRETL y guardada en formato Gretl (\texttt{*.gdt}) en el
directorio \href{IdentificaEstosARIMA/}{IdentificaEstosARIMA} (manteniendo los nombres anonimizados
pero añadiendo tras un guión la supuesta periodicidad de los datos).


\begin{minted}[frame=lines,fontsize=\scriptsize,linenos=]{python}
import itertools
import hashlib
import subprocess
import os

ar  = [ '(1 - 0.8B)', '(1 + 0.8B)', '(1 - 0.8BS)' ]
ma =  [ '(1 - 0.65B)', '(1 + 0.65B)', '(1 - 0.65BS)']
I   = [ '(1 - B)', '(1 - BS)']

def simulacion(lista_AR, lista_MA, lista_I, PeriodoEstacional=12):
    S = str(PeriodoEstacional)
    directorio = "SeriesSimuladas"+S
    d = os.path.join(directorio)
    if not os.path.exists(d):
        os.makedirs(d)

    def polinomios(monomios, S, NOlimita=1):
        polinomios = []
        for i in range(3):
            for c in itertools.combinations(monomios, i):
                cc = ''.join(c).replace("S", S)
                #polinomios = polinomios + [cc] if ((i < len(monomios)) | NOlimita) else polinomios
                polinomios = polinomios + [cc] 
        return polinomios

    SampleSize = 40*PeriodoEstacional
    numero_de_modelos = 0
    etiqueta = []
    for ca,ma in enumerate(polinomios(lista_MA,str(S),0)):
        for cb,ar in enumerate(polinomios(lista_AR,str(S),0)):
            for cc,i in enumerate(polinomios(lista_I,str(S))):
                for l in range(2):
                    if len(ar)<18 or len(ma)<18:
                        numero_de_modelos += 1 
                        parametros = "ar = '%s', ma = '%s', i = '%s'"  % (ar,ma,i)
                        modelo = "ARIMA <- um(%s)"  % (parametros)
                        if l: 
                            sim   = 'y = exp((sim(ARIMA, n = %d)) * 0.1 + 2.5)' % (SampleSize,)
                            logs  = 'logs'
                            media = 'mu = 2.5'
                        else:
                            sim   = 'y = (sim(ARIMA, n = 200))'
                            logs  = '    '
                            media = 'mu = 0.0'

                        # script de R
                        caso = chr(ca*1000+cb*100+cc*10+l)
                        label = (hashlib.shake_128(caso.encode()).hexdigest(3))
                        características = logs + ',\t' + media + ',\t' + parametros
                        etiqueta = etiqueta + [label + ',\t' + características]
                        fichero = "write.csv(y, file='%s/%s.csv')"  % (directorio,label)
                        with open('Rscript.r', 'w') as f:
                            f.write('library(tfarima)'+'\n')
                            f.write(modelo+'\n')
                            f.write(sim+'\n')
                            f.write(fichero+'\n')                            
                        subprocess.call (["/usr/bin/Rscript", "--vanilla", "./Rscript.r"])

                        # script de Gretl
                        TamañoMuestral = 'nulldata %d' % SampleSize
                        Dir          = 'string directory = $(pwd) ~ "/IdentificaEstosARIMA"'
                        DirTrabajo   = 'set workdir "@directory"'
                        periodicidad = 'setobs %s 1985:1 --time-series' % (S,)
                        datos        = 'open %s/%s.csv' % (directorio,label,)
                        guardado     = 'store %s-%s.gdt x' % (label,S)
                        Guion = "GRETLscript"+S+".inp"
                        with open(Guion, 'w') as f:
                            f.write(Dir+'\n'+DirTrabajo+'\n'+datos+'\n'+periodicidad+'\n'+guardado)
                            
                        subprocess.call (["/usr/bin/gretlcli", "-b", Guion])
                        
    print(numero_de_modelos)

    with open(directorio+'/Etiquetas.txt', 'w') as f:
        f.write('etiqueta' + ',\t' + 'logs' + ',\t' + 'mu' + ',\t' + 'ar' + ',\t' + 'ma'+',\t' + 'nabla' + '\n')
        for e in etiqueta:
            f.write(e+'\n')
    
\end{minted}

\begin{minted}[frame=lines,fontsize=\scriptsize,linenos=]{python}
ar  = [ '(1 - 0.8B)', '(1 + 0.8B)', '(1 - 0.8BS)']
ma =  [ '(1 - 0.55B)', '(1 + 0.55B)', '(1 + 0.55BS)']
I   = [ '(1 - B)', '(1 - BS)']
_ = simulacion(ar, ma, I, 4)
_ = simulacion(ar, ma, I, 12)
\end{minted}
\end{document}

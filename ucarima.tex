% Created 2024-07-16 mar 16:07
% Intended LaTeX compiler: pdflatex
\documentclass[11pt]{article}
\usepackage[utf8]{inputenc}
\usepackage{lmodern}
\usepackage[T1]{fontenc}
\usepackage[top=1in, bottom=1.in, left=1in, right=1in]{geometry}
\usepackage{graphicx}
\usepackage{longtable}
\usepackage{float}
\usepackage{wrapfig}
\usepackage{rotating}
\usepackage[normalem]{ulem}
\usepackage{amsmath}
\usepackage{textcomp}
\usepackage{marvosym}
\usepackage{wasysym}
\usepackage{amssymb}
\usepackage{amsmath}
\usepackage[theorems, skins]{tcolorbox}
\usepackage[version=3]{mhchem}
\usepackage[numbers,super,sort&compress]{natbib}
\usepackage{natmove}
\usepackage{url}
\usepackage[cache=false]{minted}
\usepackage[strings]{underscore}
\usepackage[linktocpage,pdfstartview=FitH,colorlinks,
linkcolor=blue,anchorcolor=blue,
citecolor=blue,filecolor=blue,menucolor=blue,urlcolor=blue]{hyperref}
\usepackage{attachfile}
\usepackage{setspace}
\usepackage{nacal}
\usepackage[spanish, ]{babel}
\usepackage{lmodern}
\usepackage{tabularx}
\usepackage{booktabs}
\author{Marcos Bujosa}
\date{\today}
\title{}
\begin{document}

\emph{La notación tradicional no me gusta. Usaré una más compacta con
productos convolución en lugar de sumatorios y donde
\(\Vect{y}=\{y_t\mid t\in\Zz\}\)} (ya ves que tengo cierta obsesión con
las notaciones).
\bigskip

Sea $$\Vect{y}=\Vect[1]{c}+\cdots+\Vect[k]{c}$$ donde cada
\(\Vect[j]{c}=\{{c_j}_t\mid t\in\Zz\}\) es un proceso ARIMA
$$\mathsf{A}_j*\Vect[j]{c}=\mathsf{B}_j*\Vect[j]{e},$$ es decir, donde
\(\mathsf{A}_j\) y \(\mathsf{B}_j\) son polinomios en el operador retardo
\(L\), donde \(*\) es el producto convolución y donde
\(\Vect[j]{e}=\{{e_j}_t\mid t\in\Zz\}\) es un proceso de ruido
blanco. Entonces \(\Vect{y}\) también es un proceso ARIMA
$$\mathsf{A}*\Vect{y}=\mathsf{B}*\Vect{e}$$ donde, si denotamos el
conjunto de índices \(1,2,\ldots,k\) con \(\{1:k\}\), tenemos que
$$\mathsf{A}=\mathsf{A}_1*\cdots*\mathsf{A}_k=\prod\limits_{j\in\{1:k\}}\mathsf{A}_j.$$
Consecuentemente
\begin{align*}
\mathsf{A}*\Vect{y}=&\mathsf{B}*\Vect{e}\\
\mathsf{A}*(\Vect[1]{c}+\cdots+\Vect[k]{c})=&\mathsf{B}*\Vect{e}\\
\mathsf{A}*\Vect[1]{c}+\cdots+\mathsf{A}*\Vect[k]{c}=&\mathsf{B}*\Vect{e}\\
\Parentesis*{\prod\limits_{\substack{j\in\{1:k\}\\j\ne 1}}\mathsf{A}_j}*\mathsf{A}_1*\Vect[1]{c}+
\Parentesis*{\prod\limits_{\substack{j\in\{1:k\}\\j\ne 2}}\mathsf{A}_j}*\mathsf{A}_2*\Vect[2]{c}+
\cdots+
\Parentesis*{\prod\limits_{\substack{j\in\{1:k\}\\j\ne k}}\mathsf{A}_j}*\mathsf{A}_k*\Vect[k]{c}=&\mathsf{B}*\Vect{e}\\
\Parentesis*{\prod\limits_{\substack{j\in\{1:k\}\\j\ne 1}}\mathsf{A}_j}*\mathsf{B}_1*\Vect[1]{e}+
\Parentesis*{\prod\limits_{\substack{j\in\{1:k\}\\j\ne 2}}\mathsf{A}_j}*\mathsf{B}_2*\Vect[2]{e}+
\cdots+
\Parentesis*{\prod\limits_{\substack{j\in\{1:k\}\\j\ne k}}\mathsf{A}_j}*\mathsf{B}_k*\Vect[k]{e}=&\mathsf{B}*\Vect{e}\\
\end{align*}
Y si asumimos que \(\mathsf{B}\) es invertible, entonces
$$\frac{\Parentesis*{\prod\limits_{\substack{j\in\{1:k\}\\j\ne
1}}\mathsf{A}_j}*\mathsf{B}_1}{\mathsf{B}}*\Vect[1]{e}+
\frac{\Parentesis*{\prod\limits_{\substack{j\in\{1:k\}\\j\ne
2}}\mathsf{A}_j}*\mathsf{B}_2}{\mathsf{B}}*\Vect[2]{e}+ \cdots+
\frac{\Parentesis*{\prod\limits_{\substack{j\in\{1:k\}\\j\ne
k}}\mathsf{A}_j}*\mathsf{B}_k}{\mathsf{B}}*\Vect[k]{e}=\Vect{e}.$$

Me da en la nariz que la condición no puede ser exclusivamente sobre
el polinomio \(\mathsf{B}\), probablemente también debe haber
condiciones sobre los procesos de ruido blanco \(\Vect[1]{e}\),
\(\Vect[2]{e},\ldots\) \(\Vect[k]{e}\), pues la suma de los \(k\) procesos
de la izquierda en la última expresión resulta ser ruido blanco.
\bigskip

No sé ¿voy bien orientado?\ldots{} ¿o toda esta deducción es una
gilipollez?
\end{document}
